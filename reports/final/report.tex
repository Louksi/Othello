\documentclass[a4paper,12pt]{article}
\usepackage[utf8]{inputenc}
\usepackage[T1]{fontenc}
\usepackage[french]{babel}
\usepackage{csquotes}
\usepackage{geometry}
\usepackage{graphicx}
\usepackage{listings}
\usepackage{hyperref}
\usepackage{float}
\usepackage{appendix}
\usepackage[backend=biber]{biblatex}
\geometry{top=2cm, bottom=2cm, left=2cm, right=2cm}

\addbibresource{bibliography.bib}

\begin{document}

\begin{titlepage}
    \centering
    \vspace*{1cm}
    {\huge\bfseries Jeu de plateau: Othello \par}
    \vspace{3cm}

    {Matis Duval, Rémy Heuret, Lucas Marques, Gabriel Tardiou \par}
    \vspace{2cm}

    {\scshape\small Université de Bordeaux \par}
    \vspace{1cm}
    {\scshape\small Projet de Programmation, Master 1 \par}
    \vspace{1cm}

    \vfill
    {\large Avril 2025 \par}
\end{titlepage}

\newpage

\tableofcontents

\newpage

\section{Présentation de l'existant}

\begingroup

Le jeu d’Othello que l’on connaît aujourd’hui est une version modifiée du jeu
Reversi, créé autour de 1880, supposément par Lewis Waterman ou John W.
Mollett. Reversi lui-même étant une variante du jeu Annexation créé en 1870 par
J. W. Mollett – la seule différence étant le plateau de jeu: une croix de 10
par 4 pour Annexation, et un carré de 8 par 8 pour Reversi. Ces deux personnes
d’origine anglaise se disputent l’invention du jeu, et ont créé deux versions
différentes : Reversi Waterman et Reversi Mollett.\par

\begin{figure}[h]
    \centering
    \includegraphics[width=0.7\textwidth]{example-image} % Replace with your image file
    \caption{Jeu Reversi, version Lewis Waterman.\\
        Le jeu comporte un papier représentant le plateau, un livret de règles, et des pions.\\
        Image prise sur le site de la Fédération Française d’Othello.\\
        \url{https://www.ffothello.org/images/histoire/jeux-anciens/reversi_waterman-1880.jpg}}
    \title{fig:Jeu Reversi, version Lewis Waterman.}
\end{figure}

\begin{figure}[h]
    \centering
    \includegraphics[width=0.7\textwidth]{example-image} % Replace with your image file
    \caption{Jeu Reversi, version J. W. Mollett.\\
        Le jeu comporte un plateau en papier et des pions.\\
        Image prise sur le site de la Fédération Française d’Othello.\\
        \url{https://www.ffothello.org/images/histoire/jeux-anciens/reversi_mollett-1880.jpg}
    }
    \title{fig:Jeu Reversi, version J. W. Mollett.}
\end{figure}

\noindent Ce jeu de plateau était très apprécié vers la fin du 19\up{ème} siècle, surtout en
Angleterre. Un article sur le jeu parut en 1888 dans un magazine spécialisé
dans les jeux de dames. On retrouve des éditions aux États-Unis, ainsi qu’en
Europe Centrale, et des livres de stratégie sont créés.\par

\noindent Le jeu Reversi perd de sa popularité au 20\up{ème} siècle, jusqu’en 1971, où un
Japonais du nom de Goro Hasegawa réinvente et redistribua le jeu sous un autre
nom : Othello. Le père de Goro Hasegawa, un professeur de littérature anglaise,
lui proposa le nom d’Othello en référence à la pièce de W. Shakespeare, en
raison des nombreux retournements de situation.\par

\noindent Le jeu devient rapidement populaire, et la première compétition est organisée
en 1973, soit deux ans après la commercialisation du jeu.\par

\noindent Dès 1976, Othello est arrivé en Angleterre et aux États-Unis, et les premiers
championnats du monde d’Othello se tiennent en 1977, et reviennent tous les
ans.\par

\noindent Les règles d’Othello diffèrent légèrement de celles de Reversi. Désormais, on
fixe la position initiale des pions, et il est possible de prendre des pions à
son adversaire lorsque celui-ci passe son tour.\par

\begin{figure}[h]
    \centering
    \includegraphics[width=0.7\textwidth]{example-image} % Replace with your image file
    \caption{Jeu Othello, Othello Royal distribué par Tsukada.\\
        Jeu utilisé en tournois en France.\\
        Image prise sur le site de la Fédération Française d’Othello.\\
        \url{https://www.ffothello.org/images/histoire/jeux-modernes/othelloroyal.jpg}
    }
    \title{fig:Jeu Reversi, version J. W. Mollett.}
\end{figure}

\noindent En France, le jeu se popularise à partir de la fin des années 1970, et la
Fédération Française d’Othello (FFO) est créée en 1983.\par

\hfill \break
\endgroup

\begingroup

Afin d’établir des stratégies, les joueurs doivent avoir en tête certains
concepts clés. Tout d’abord, il faut se rappeler que le nombre de pièces de
chaque joueur peut changer rapidement au cours de la partie. Il est important
de placer des pierres stables, qui ne pourront pas être capturées.\par

\noindent Ensuite, placer ses pions dans les coins garantit souvent les pièces autour
comme stables. Le joueur essaiera de ne pas jouer dans les cases proches des
coins pour ne pas les donner à son adversaire.\par

\noindent Il faut essayer de réduire le plus possible les options de son adversaire, tout
en essayant d’avoir beaucoup de choix de son côté.\par

\noindent Calculer le nombre de cases restant permet de se projeter sur la fin de partie,
et de qui des deux joueurs va placer le dernier pion. Comme le résultat est
calculé à partir du dernier état du jeu, jouer en dernier peut faire une
différence. On peut calculer ses prochains coups en conséquence, essayer de
faire passer son tour à son adversaire, ou l’inciter à jouer sur certaines
cases.\par

\hfill \break
\endgroup

\begingroup

Les règles en tournoi, spécifiquement, pour les championnats du monde d’Othello
(World Othello Championships – WOC) sont décrites dans le document World
Othello Championships Rules, publié en septembre 2019 par la Fédération
Mondiale d’Othello (World Othello Federation – WOF).\par

\noindent Les championnats du monde se tiennent annuellement, et déterminent le champion
des catégories Individuels, Femmes, et Jeunes.\par

La compétition se tient sur trois jours. Pendant les deux premiers jours, le
rang est déterminé par des matchs à temps limité, en rondes Suisses ou Robin,
puis les demi-finales et la finale se tiennent le troisième jour.\par

\noindent Seulement les équipes des nations membres de la WOF peuvent participer.\par

Les règles sont très spécifiques quant à l’éligibilité des joueurs, le compte
des scores, et l’organisation du tournoi. Les procédures pour l’édition des
listes de classement, ou en cas de matériel défectueux, d’égalité y sont
également décrites très précisément.\par

\noindent Le site de la Fédération Mondiale d’Othello tient à jour un calendrier
\footnote{Calendrier international des tournois d'Othello, régulièrement mis à jour par la Fédération Mondiale d'Othello. \url{https://www.worldothello.org/calendar}} des
prochains tournois organisés.\par

\noindent Des championnats européens se tiendront le 31 mai et 1er juin de cette année, à
Prague en République Tchèque.\par

\noindent Le prochain championnat du monde aura lieu à Ankara en Turquie, en novembre
2025, et réunira plus de 84 pays.\par

\hfill \break
\endgroup

\begingroup

Le jeu d’Othello est relativement populaire, on en trouve fréquemment en clubs
de jeux de société ou jeux de plateau.\par

Il est également possible de jouer en ligne sur des sites, recommandés par la
FFO \footnote{Guide des plateformes en ligne pour jouer à Othello, recommandé
    par la Fédération Française d'Othello.
    \url{https://www.ffothello.org/communaute/jouer-sur-internet/}}. Les plus
populaires cités sont PlayOK \footnote{Site PlayOK offrant la possibilité de
    jouer au Reversi en ligne contre des adversaires du monde entier, avec système
    de classement. \url{https://www.playok.com/fr/reversi/}}, qui propose des
parties en ligne ; et eOthello \footnote{Plateforme eOthello permettant de
    jouer gratuitement à Othello en différé. \url{https://www.eothello.com/}} pour
jouer plusieurs parties en différé, avec 72 heures de temps limite par coup.\par

Le site PlayPager \footnote{Plateforme PlayPager permettant de jouer
    gratuitement à Othello/Reversi en ligne contre l'ordinateur ou d'autres
    joueurs. \url{https://playpager.com/othello-reversi/}} permet de jouer à deux
joueurs ou contre une intelligence artificielle sur un plateau en 8 par 8.\par

\noindent La Fédération Française d’Othello propose également de se mettre en relation
avec des joueurs via une liste de diffusion.\par

\noindent Il existe aussi des serveurs discord tels que Othello Academy \footnote{Serveur Discord "Othello Academy" regroupant une communauté active de joueurs, des tutoriels et des tournois amicaux. \url{https://discord.me/othelloacademy}}, ou encore Board
Games Geek.\par

\hfill \break
\endgroup

\newpage

\section{Règles du jeu}

\section{Descriptions des algorithmes et structures de données}

\newpage

\section{Architecture du projet}

\newpage

\section{Performances et limitations}

\section{Critiques et perspectives}

\newpage

\section{Bibliographie}

\newpage

\begin{appendices}

    \section{Liste des besoins étendus}

    \section{Liste des besoins non réalisés}

    \section{Agenda rétrospectif}

\end{appendices}
\newpage

% Example of how to insert code
\begin{lstlisting}[language=C, caption=Exemple d'insertion de code]
    // Voici un exemple de code en C
    #include <stdio.h>

    int main() {
        printf("Hello, Othello!\n");
        return 0;
    }
\end{lstlisting}

% Example of how to insert an image
\begin{figure}[h]
    \centering
    \includegraphics[width=0.7\textwidth]{example-image} % Replace with your image file
    \caption{Exemple d'insertion d'image}
    \title{fig:example-image}
\end{figure}

% Example of a footnote
Voici un exemple de texte avec une note de bas de page\footnote{Ceci est un
    exemple de note de bas de page.}.

% Example of citation
Ceci est un exemple de citation \citation{Othello1980}.

\end{document}

