\documentclass[a4paper,12pt]{article}
\usepackage[french]{babel}
\usepackage{hyperref}
\usepackage{csquotes}
\usepackage{geometry}
\usepackage{graphicx}
\usepackage{listings}
\usepackage{hyperref}
\usepackage{float}
\usepackage{appendix}
\usepackage{xcolor}
\usepackage{colortbl}
\usepackage{tabularx}
\usepackage{array}
\usepackage[utf8]{inputenc}
\usepackage[T1]{fontenc}
\usepackage[numbers]{natbib}
\usepackage{placeins} % for \FloatBarrier because of the images in Agenda Rétrospectif
\geometry{top=2cm, bottom=2cm, left=2cm, right=2cm}

\definecolor{CornflowerBlue}{RGB}{100,149,237}  % For functional needs
\definecolor{RoyalBlue}{RGB}{65,105,225}   % For non-functional needs
\definecolor{MediumAquamarine}{RGB}{102,205,170} % For validation strategy

\begin{document}

\begin{titlepage}
    \centering
    \vspace*{1cm}
    {\huge\bfseries Jeu de plateau: Othello}
    \vspace{3cm}

    {Matis Duval, Rémy Heuret, Lucas Marques, Gabriel Tardiou}
    \vspace{2cm}

    {\scshape\small Université de Bordeaux\\}
    \vspace{1cm}
    {\scshape\small Projet de Programmation, Master 1}
    \vspace{1cm}

    \vfill
    {\large Avril 2025}
\end{titlepage}

\newpage

\tableofcontents

\newpage

\section{Présentation de l'existant}

Le jeu d’Othello que l’on connaît aujourd’hui est une version modifiée du jeu
Reversi, créé autour de 1880, supposément par Lewis Waterman ou John W.
Mollett.\cite{FFOthelloHistoire} Reversi lui-même étant une variante du jeu
Annexation créé en 1870 par J. W. Mollett – la seule différence étant le
plateau de jeu: une croix de 10 par 4 pour
Annexation\cite{TameTheBoardGameAnnexation}, et un carré de 8 par 8 pour
Reversi. Ces deux personnes d’origine anglaise se disputent l’invention du jeu,
et ont créé deux versions différentes : Reversi Waterman\footnote{Image prise
    sur le site de la Fédération Française d’Othello.
    \url{https://www.ffothello.org/images/histoire/jeux-anciens/reversi_waterman-1880.jpg}}
et Reversi Mollett\footnote{Image prise sur le site de la Fédération Française
    d’Othello.
    \url{https://www.ffothello.org/images/histoire/jeux-anciens/reversi_mollett-1880.jpg}}.\\

\begin{figure}[h]
    \centering
    \includegraphics[width=0.5\textwidth]{images/reversi_waterman-1880.jpg}
    \caption{Jeu Reversi, version Lewis Waterman .\\
        Le jeu comporte un papier représentant le plateau, un livret de règles, et des pions.}
    \title{fig:Jeu Reversi, version Lewis Waterman.}

    \centering
    \includegraphics[width=0.5\textwidth]{images/reversi_mollett-1880.jpg}
    \caption{Jeu Reversi, version J. W. Mollett.\\
        Le jeu comporte un plateau en papier et des pions.}
    \title{fig:Jeu Reversi, version J. W. Mollett.}
\end{figure}

Ce jeu de plateau était très apprécié vers la fin du 19\up{ème} siècle, surtout
en Angleterre. Un article sur le jeu parut en 1888 dans un magazine spécialisé
dans les jeux de dames.\\ On retrouve des éditions aux États-Unis, ainsi qu’en
Europe Centrale, et des livres de stratégie sont créés.\\ Le jeu Reversi perd
de sa popularité au 20\up{ème} siècle, jusqu’en 1971, où un Japonais du nom de
Goro Hasegawa réinvente et redistribue le jeu sous un autre nom : Othello. Le
père de Goro Hasegawa, un professeur de littérature anglaise, lui propose le
nom d’Othello en référence à la pièce de W. Shakespeare, en raison de ses
nombreux retournements de situation.\\ Le jeu devient rapidement populaire, et
la première compétition est organisée en 1973, soit deux ans après la
commercialisation du jeu.\\ Dès 1976, Othello est arrivé en Angleterre et aux
États-Unis, et les premiers championnats du monde d’Othello se tiennent en
1977, et reviennent tous les ans.\\ Les règles d’Othello diffèrent légèrement
de celles de Reversi. Désormais, on fixe la position initiale des pions, et il
est possible de prendre des pions (non placés sur le plateau) à son adversaire
lorsque celui-ci passe son tour.\\ En France, le jeu se popularise à partir de
la fin des années 1970, et la Fédération Française d’Othello (FFO) est créée en
1983\cite{FFOthelloTournois}\footnote{Image prise sur le site de la Fédération
    Française d’Othello.
    \url{https://www.ffothello.org/images/histoire/jeux-modernes/othelloroyal.jpg}}.\\

\begin{figure}[h]
    \centering
    \includegraphics[width=0.5\textwidth]{images/othelloroyal.jpg}
    \caption{Jeu Othello, Othello Royal distribué par Tsukada.\\
        Jeu utilisé en tournois en France.}
    \title{fig:Jeu Reversi, version J. W. Mollett.}
\end{figure}

Afin d’établir des stratégies, les joueurs doivent avoir en tête certains
concepts clés.\cite{FFOthelloStrategie}\cite{FFOthelloOuvertures} Tout d’abord,
il faut se rappeler que le nombre de pièces de chaque joueur peut changer
rapidement au cours de la partie. Il est important de placer des pierres
stables, qui ne pourront pas être capturées.\\ Ensuite, placer ses pions dans
les coins garantit souvent les pièces autour comme stables. Le joueur essaiera
de ne pas jouer dans les cases proches des coins pour ne pas les donner à son
adversaire.\\ Il faut tenter de réduire le plus possible les options de son
adversaire, tout en essayant d’avoir beaucoup de choix de son côté.\\ Calculer
le nombre de cases restantes permet de se projeter sur la fin de partie, et de
qui des deux joueurs va placer le dernier pion. Comme le résultat est calculé à
partir du dernier état du jeu, jouer en dernier peut faire une différence. On
peut calculer ses prochains coups en conséquence, essayer de faire passer son
tour à son adversaire, ou l’inciter à jouer sur certaines cases.\\

Les règles en tournoi, spécifiquement, pour les championnats du monde d’Othello
(World Othello Championships – WOC) sont décrites dans le document World
Othello Championships Rules, publié en septembre 2019 par la Fédération
Mondiale d’Othello (World Othello Federation – WOF).\cite{WOCRules2023}\\ Les
championnats du monde se tiennent annuellement, et déterminent le champion des
catégories Individuels, Femmes, et Jeunes.\\ La compétition se tient sur trois
jours. Pendant les deux premiers jours, le rang est déterminé par des matchs à
temps limité, en rondes Suisses ou Robin, puis les demi-finales et la finale se
tiennent le troisième jour.\\ Seulement les équipes des nations membres de la
WOF peuvent participer.\\ Les règles sont très spécifiques quant à
l’éligibilité des joueurs, le compte des scores, et l’organisation du tournoi.
Les procédures pour l’édition des listes de classement, ou en cas de matériel
défectueux, d’égalité y sont également décrites très précisément.\\ Le site de
la Fédération Mondiale d’Othello tient à jour un calendrier
\footnote{Calendrier international des tournois d'Othello, régulièrement mis à
    jour par la Fédération Mondiale d'Othello.
    \url{https://www.worldothello.org/calendar}} des prochains tournois
organisés.\cite{WOFCalendar}\cite{WOCTournaments}\\ Des championnats européens
se tiendront le 31 mai et 1er juin de cette année, à Prague en République
Tchèque.\\ Le prochain championnat du monde aura lieu à Ankara en Turquie, en
novembre 2025, et réunira plus de 84 pays.\\

Le jeu d’Othello est relativement populaire, on en trouve fréquemment en clubs
de jeux de société ou jeux de plateau.\\ Il est également possible de jouer en
ligne sur des sites, recommandés par la FFO \footnote{Guide des plateformes en
    ligne pour jouer à Othello, recommandé par la Fédération Française d'Othello.
    \url{https://www.ffothello.org/communaute/jouer-sur-internet/}}.\cite{FFOthelloJouerInternet}
Les plus populaires cités sont PlayOK\cite{PlayOKReversi} \footnote{Site PlayOK
    offrant la possibilité de jouer au Reversi en ligne contre des adversaires du
    monde entier, avec système de classement.
    \url{https://www.playok.com/fr/reversi/}}, qui propose des parties en ligne ;
et eOthello \footnote{Plateforme eOthello permettant de jouer gratuitement à
    Othello en différé. \url{https://www.eothello.com/}} pour jouer plusieurs
parties en différé, avec 72 heures de temps limite par coup.\\ Le site
PlayPager\cite{PlayPagerOthello} \footnote{Plateforme PlayPager permettant de
    jouer gratuitement à Othello/Reversi en ligne contre l'ordinateur ou d'autres
    joueurs. \url{https://playpager.com/othello-reversi/}} permet de jouer à deux
joueurs ou contre une intelligence artificielle sur un plateau en 8 par 8.\\ La
Fédération Française d’Othello propose également de se mettre en relation avec
des joueurs via une liste de diffusion.\\ Il existe aussi des serveurs discord
tels que Othello Academy \footnote{Serveur Discord "Othello Academy" regroupant
    une communauté active de joueurs, des tutoriels et des tournois amicaux.
    \url{https://discord.me/othelloacademy}}, ou encore Board Games Geek sur
lequels des joueurs peuvent se rencontrer, discuter du jeu et de stratégie.\\

\newpage

\section{Règles du jeu}

Othello se joue sur un plateau unicolore de 64 cases : l’\textit{othellier}. \\
Les cases de l’othellier sont notées avec les lettres de A à H pour les colonnes et les chiffres de 1 à 8 pour les lignes. \\
Chaque joueur possède 64 pierres, de couleur blanche sur une face et noire sur l’autre.

\vspace{0.4cm}

Avant que la partie ne commence, 4 pierres sont déposées sur l'othellier : 2 pierres noires en \texttt{E4} et \texttt{D5}, et 2 pierres blanches en \texttt{E5} et \texttt{D4}.

\begin{figure}[h]
    \centering
    \includegraphics[width=0.5\textwidth]{images/othello_rules_start.png}
    \caption{Position de départ de l'othellier.}
    \title{fig:Début de partie.}
\end{figure}


\vspace{0.4cm}

Le but du jeu est de terminer la partie en ayant plus de pierres de sa couleur que celle de l’adversaire sur l’othellier. \\
Au début de la partie, les joueurs noir et blanc sont désignés. Chaque joueur doit poser ses pierres sur l’othellier avec la face de sa couleur visible.

\vspace{0.4cm}

Les joueurs jouent à tour de rôle, en commençant par le joueur noir. \\
Un tour désigne une pierre posée de sorte à capturer au minimum une pierre adverse.

\vspace{0.4cm}

Pour capturer une (ou plusieurs) pierres, il faut positionner de part et d’autre des pierres adverses une pierre de sa propre couleur, sur une même ligne, colonne ou diagonale. \\
Toutes les pierres ennemies capturées sont alors retournées, pour que la face visible soit de la couleur du joueur qui a initié la capture. \\

\vspace{0.4cm}

Il faut noter qu’une capture ne peut être initiée qu’à partir de la pierre qui vient d’être posée. \\
Il ne peut donc pas y avoir de réaction en chaîne. \\
Si la pierre posée permet de capturer dans plusieurs directions, alors toutes les pierres doivent être capturées, peu importe la direction.

\vspace{0.4cm}

Voici un exemple de capture : \\

Les noirs souhaitent jouer le coup \texttt{C6}. On regarde alors dans toutes les directions en partant de cette pierre s'il y a des pierres blanches prises en "sandwich" par pierres noires.
Les pierres \texttt{C5} et \texttt{D5} sont donc les 2 pierres capturées.

\begin{figure}[h]
    \centering
    \includegraphics[width=0.5\textwidth]{images/othello_rules_before_move.png}
    \caption{Coups possibles par le joueur noir.}
    \title{fig:Coups possibles noirs.}
\end{figure}

\begin{figure}[h]
    \centering
    \includegraphics[width=0.5\textwidth]{images/othello_rules_after_move.png}
    \caption{Capture des pierres \texttt{C5} et \texttt{D5} par le joueur noir.}
    \title{fig:Capture noire.}
\end{figure}

La partie peut se terminer de deux manières :
\begin{itemize}
    \item L’othellier est complètement rempli (64 pierres posées).
    \item Les deux joueurs n’ont plus de coup possible.
\end{itemize}

\vspace{0.2cm}

Une fois la partie terminée, on compte le nombre de pierres possédées par chaque joueur. Le joueur possédant le plus de pierres remporte la partie. \\

Les règles se fient à celles de la FFO (Fédération Française d'Othello \cite{FFOthelloRegles}).

\section{Descriptions des algorithmes et structures de données}

\newpage

\section{Architecture du projet}

Nous avons décidé d’adopter une architecture MVC. Ici, nous commencerons par
rappeler brièvement ce que cela implique et comment cela se traduit dans notre
implémentation, avant de détailler le cycle de vie de notre application.\\

\textbf{Le Modèle}\\

Notre application repose sur notre implémentation de \texttt{Bitboard}. En
effet, le jeu d’Othello se prête particulièrement bien à l’utilisation des
\texttt{bitboards} étant donné que l’on peut représenter un état du jeu avec
seulement deux \texttt{bitboards}, un par couleur de joueur.\\ Celle-ci n’est
pas spécifique au jeu d’Othello, mais constitue une implémentation classique de
\texttt{bitboards} tirant parti des particularités du langage \texttt{Python},
telles que l’utilisation par défaut d’entiers de taille arbitraire et la
surcharge d’opérateurs.\\ Pour des raisons de performance, certains attributs
-- comme les masques utilisés -- sont calculés une fois pour toutes, car leur
génération dynamique à chaque instanciation d’un \texttt{bitboard} serait
excessivement coûteuse.\\

Ces \texttt{bitboards} sont directement utilisés par notre modèle,
\texttt{OthelloBoard}, qui gère l’intégralité de l’état d’un plateau de jeu.
Lors de son initialisation, il reçoit une taille issue de l’énumération
\texttt{BoardSize}, qui nous permet également de retrouver un
\texttt{BoardSize} en fonction d’un entier, si celui-ci est une taille de board
valide (6, 8, 10 ou 12).\\ Ce modèle gère ainsi non seulement les deux
\texttt{bitboards} de pions (un pour les noirs, un pour les blancs), mais aussi
l’implémentation des deux algorithmes principaux nécessaires au jeu~:\\
\texttt{line\_cap\_move}~: qui génère un \texttt{bitboard} représentant les
coups possibles pour un joueur donné à partir de l’état actuel du jeu.\\
\texttt{line\_cap}~: qui génère un \texttt{bitboard} des pions capturés à
partir d’un coup (légal) et d’un joueur.\\

Ces algorithmes n’induisent aucune mutation sur le \texttt{bitboard} courant.
Initialement, ils s’appuyaient sur la génération de nombreux \texttt{bitboards}
intermédiaires – ce qui permettait, grâce à la surcharge d’opérateurs,
d’exprimer élégamment les algorithmes.\\ Toutefois, lors d’un
\textit{profiling} visant à réduire la lenteur de l’exploration des arbres de
jeu pour l’IA, nous avons constaté un goulot d’étranglement dû à la création
excessive de ces objets intermédiaires.\\ C’est pourquoi nous avons décidé de
remplacer la boucle de \texttt{line\_cap\_move} par une version déroulée, et
d’optimiser la création ainsi que la copie des \texttt{bitboards}.\\

Pour jouer un coup, c’est-à-dire pour appliquer une mutation sur l’état du
plateau, il faut passer par la fonction \texttt{play()}. Celle-ci reçoit des
coordonnées et, si le coup est possible, le joue pour le joueur courant.\\ En
plus de mettre à jour le plateau, la fonction se charge de changer le joueur
courant, tout en vérifiant si le nouveau joueur courant dispose d’un coup
jouable. Si ce n’est pas le cas, elle enregistre dans l’historique
l’impossibilité pour le nouveau joueur de jouer et rétablit le joueur
initial.\\

Si le joueur initial n’a toujours pas de coup à jouer, le plateau se trouve en
état de « game over ». Même si cela n’implique rien, nous lancions auparavant
une exception, mais celle-ci n’était finalement pas utile puisque vérifiable à
tout moment via \texttt{is\_game\_over()}, ce qui est plus simple que
d’attendre une exception.\\

\newpage

Par ailleurs, \texttt{OthelloBoard} conserve l’historique des coups joués,
permettant, grâce à\\ \texttt{get\_last\_play()}, de retrouver le dernier coup
effectué ou, avec \texttt{pop()}, de retirer ce dernier coup s'il existe. Cette
fonctionnalité a été ajoutée pour faciliter l’exploration des arbres de jeu
dans le cadre de l’implémentation des joueurs IA, afin d’éviter de cloner
l’état avant chaque mutation.\\

Nous obtenons ainsi une modélisation évolutive d’un plateau d’Othello,
respectant les règles de placement de pions, sans pour autant implémenter
d’autres fonctionnalités de base.\\

\textbf{Le Contrôleur}\\

Passons maintenant au contrôleur. \texttt{GameController} se construit à partir
d’un \texttt{OthelloBoard} déjà configuré et de deux joueurs (objets issus
d’une classe dérivée de \texttt{Player}). Il reçoit également les paramètres du
mode blitz : un booléen et un entier indiquant le temps total de jeu si ce mode
est activé.\\ Lors de son instanciation, il crée les deux joueurs en leur
assignant la couleur adéquate (le joueur noir étant créé en premier, le joueur
blanc en second). Il s’enregistre aussi auprès de chaque joueur afin que
ceux-ci puissent appeler la fonction \texttt{play()} du contrôleur lors de leur
action.\\

La méthode \texttt{play()} prend deux coordonnées, \texttt{x\_coord} et
\texttt{y\_coord}, et exécute les étapes suivantes~: - Vérifier que la partie
est toujours en cours en consultant l’attribut \texttt{game\_over} (ce dernier
est mis à \texttt{True} dès que la fin de partie est détectée).\\ - S’assurer
que, dans le mode blitz, le timer du joueur courant n’a pas atteint zéro
(sinon, le joueur perd).\\ - Jouer le coup sur le \texttt{OthelloBoard}.\\ -
Appeler, le cas échéant, un callback \texttt{post\_play\_callback()} (par
exemple, pour actualiser l’affichage dans la CLI).\\ - Changer le joueur
courant pour la gestion du timer de blitz.\\

Cette méthode est invoquée par les vues lorsqu’un joueur humain est en capacité
de jouer.\\

\textbf{Les Joueurs}\\

La classe \texttt{Player} représente un joueur et lui associe un contrôleur et
une couleur – indispensable pour utiliser des méthodes comme
\texttt{line\_cap\_move()}.\\ Le contrôleur permet quant à lui d’accéder à
\texttt{line\_cap\_move} et de jouer le coup. Pour un joueur humain, la méthode
\texttt{call\_human\_callback()} est appelée lorsqu’il doit jouer.\\ Dans les
faits, cela n’est utilisé que pour le CLI, la GUI se reposant sur
\texttt{current\_player\_is\_human()} du contrôleur pour savoir si le joueur
courant est un humain, méthode appelée à chaque clic sur le board.\\

\textbf{Les Vues}\\

Les vues sont chargées de l’affichage : soit en mode texte dans la CLI, soit
via une fenêtre GTK4 dans la GUI, et elles gèrent également la boucle de jeu.
\\ Bien que le contrôleur détermine l’état de la partie (par exemple, en
bloquant l’action en cas de « game over »), ce sont les vues qui envoient les
actions. La principale différence réside dans la gestion des entrées : en CLI,
l’utilisateur est sollicité de manière bloquante, tandis qu’en GUI, c’est le
clic de l’utilisateur qui déclenche l’action.\\ À chaque clic, la vue vérifie
si le contrôleur attend un coup de la part d’un joueur humain et, si c’est le
cas, le joue. \\ De plus, dans le cas de la GUI, la boucle de jeu est gérée par
GTK via des callbacks, contrairement à la CLI où le contrôleur appelle
directement la méthode \texttt{next\_move()} dans sa boucle de jeu.\\

\textbf{Cycle de Vie de l’Application}\\

Le point d’entrée de l’application est \texttt{\_\_main\_\_.py}. Celui-ci
commence par récupérer un objet de configuration (\texttt{Config}) via
\texttt{argparse}.\\ Si un nom de fichier est fourni, il tente de charger cet
état dans un \texttt{OthelloBoard}, sinon, il initialise un
\texttt{OthelloBoard} avec la taille indiquée en argument ou par défaut.\\
Ensuite, les deux joueurs (noir et blanc) sont créés, en tenant compte de la
présence ou non d’un joueur IA.\\ Le contrôleur est ensuite initialisé en
fonction du mode blitz, puis transmis à la vue adéquate, laquelle lance la
boucle de jeu (pour la CLI) ou le démarrage de GTK (pour la GUI).\\

\newpage

\section{Performances et limitations}

\section{Critiques et perspectives}

\newpage

\appendix
\section{Liste des besoins étendus}

% For a functional need - border in CornflowerBlue
\noindent
\setlength{\arrayrulewidth}{1.5pt}
\renewcommand{\arraystretch}{1.5}
\arrayrulecolor{CornflowerBlue}
\begin{tabularx}{\textwidth}{|X|}
    \hline
    \textbf{F.xx NOM DU BESOIN} \\
    \hline
    Besoin fonctionnel          \\
    \hline
    Terminé le DATE             \\
    Par NOM                     \\
    \hline
    Explication du              \\
    besoin                      \\
    \arrayrulecolor{MediumAquamarine}\hline
    \arrayrulecolor{CornflowerBlue}
    Stratégie de validation     \\
    \arrayrulecolor{MediumAquamarine}\hline
    \arrayrulecolor{CornflowerBlue}
    Dépendances                 \\
    \hline
\end{tabularx}

\vspace{1cm}

% For a non-functional need (RoyalBlue)
\noindent
\setlength{\arrayrulewidth}{1.5pt}
\renewcommand{\arraystretch}{1.5}
\arrayrulecolor{RoyalBlue}
\begin{tabularx}{\textwidth}{|X|}
    \hline
    \textbf{F.xx NOM DU BESOIN} \\
    \hline
    Besoin non fonctionnel      \\
    \hline
    Terminé le DATE             \\
    Par NOM                     \\
    \hline
    Explication du              \\
    besoin                      \\
    \arrayrulecolor{MediumAquamarine}\hline
    \arrayrulecolor{RoyalBlue}
    Stratégie de validation     \\
    \arrayrulecolor{MediumAquamarine}\hline
    \arrayrulecolor{RoyalBlue}
    Dépendances                 \\
    \hline
\end{tabularx}

\vspace{1cm}

\newpage

\noindent
\setlength{\arrayrulewidth}{1.5pt}
\renewcommand{\arraystretch}{1.5}
\arrayrulecolor{RoyalBlue}
\begin{tabularx}{\textwidth}{|X|}
    \hline
    \textbf{F.1 Langage de programmation}          \\
    \hline
    Besoin non fonctionnel                         \\
    \hline
    Terminé le 06/02                               \\
    Par Lucas                                      \\
    \hline
    Développer le logiciel en langage Python 3.7+. \\
    \hline
\end{tabularx}

\vspace{1cm}

\noindent
\setlength{\arrayrulewidth}{1.5pt}
\renewcommand{\arraystretch}{1.5}
\arrayrulecolor{RoyalBlue}
\begin{tabularx}{\textwidth}{|X|}
    \hline
    \textbf{F.2 - Style de codage}                                                          \\
    \hline
    Besoin non fonctionnel                                                                  \\
    \hline
    Terminé le 04/02                                                                        \\
    Par Lucas                                                                               \\
    \hline
    Le code du programme suit le coding style du PEP8.                                      \\
    L’utilisation de pre-commit hooks permet de s’assurer du respect de ce style de codage. \\
    \hline
\end{tabularx}

\vspace{1cm}

\noindent
\setlength{\arrayrulewidth}{1.5pt}
\renewcommand{\arraystretch}{1.5}
\arrayrulecolor{RoyalBlue}
\begin{tabularx}{\textwidth}{|X|}
    \hline
    \textbf{F.3 - Langue par défaut du code}                                                                            \\
    \hline
    Besoin non fonctionnel                                                                                              \\
    \hline
    Terminé le 02/02                                                                                                    \\
    Par Matis, Rémy, Lucas, Gabriel                                                                                     \\
    \hline
    Le code, la documentation ainsi que les fichiers annexes sont écrits et commentés intégralement en langue anglaise. \\
    \hline
\end{tabularx}

\vspace{1cm}

\noindent
\setlength{\arrayrulewidth}{1.5pt}
\renewcommand{\arraystretch}{1.5}
\arrayrulecolor{RoyalBlue}
\begin{tabularx}{\textwidth}{|X|}
    \hline
    \textbf{F.4 - Système cible
    }                                                                                               \\
    \hline
    Besoin non fonctionnel                                                                          \\
    \hline
    Terminé le 06/02                                                                                \\
    Par Lucas Zammit                                                                                \\
    \hline
    Développement du logiciel sous une distribution Linux.                                          \\
    L’intégration de Python rend compliqué voir impossible l’installation de modules
    python sur l’environnement par défaut.                                                          \\
    Sur les systèmes Debian récents, il est donc nécessaire de passer par un environnement virtuel. \\
    \arrayrulecolor{MediumAquamarine}\hline
    \arrayrulecolor{RoyalBlue}
    Tester sous une distribution Linux Debian, et une distribution Ubuntu.                          \\
    \hline
\end{tabularx}

\vspace{1cm}

\noindent
\setlength{\arrayrulewidth}{1.5pt}
\renewcommand{\arraystretch}{1.5}
\arrayrulecolor{RoyalBlue}
\begin{tabularx}{\textwidth}{|X|}
    \hline
    \textbf{F.5 - Documentation}                                                                                                                                                                                                                                           \\
    \hline
    Besoin non fonctionnel                                                                                                                                                                                                                                                 \\
    \hline
    Terminé le DATE                                                                                                                                                                                                                                                        \\
    Par Matis, Rémy, Lucas, Gabriel                                                                                                                                                                                                                                        \\
    \hline
    Une documentation Sphinx automatisée permet de rendre compte des différentes fonctions, leurs attributs et leur utilité pour chaque module.                                                                                                                            \\
    Le code est documenté à l’aide de docstrings et commentaires.                                                                                                                                                                                                          \\
    Un manuel utilisateur, ainsi qu’un manuel utilisateur pour les développeurs sont disponibles. Ils proposent des explications pour l’installation et l’utilisation du logiciel d’un point de vue utilisateur, et en proposant plus de précisions pour les développeurs. \\
    Une option d’aide est présente en dehors d’une partie, afin d’indiquer à l’utilisateur comment lancer une partie et quelles options sont à sa disposition.                                                                                                             \\
    En cours de partie, une option d’aide et une présentation des règles du jeu est également disponible.                                                                                                                                                                  \\
    \hline
\end{tabularx}

\vspace{1cm}

\noindent
\setlength{\arrayrulewidth}{1.5pt}
\renewcommand{\arraystretch}{1.5}
\arrayrulecolor{RoyalBlue}
\begin{tabularx}{\textwidth}{|X|}
    \hline
    \textbf{F.6 - Tests}                                                                                                                                                            \\
    \hline
    Besoin non fonctionnel                                                                                                                                                          \\
    \hline
    Terminé le DATE                                                                                                                                                                 \\
    Par Matis, Rémy, Lucas, Gabriel                                                                                                                                                 \\
    \hline
    Les fonctionnalités du logiciel sont testées automatiquement. Il est possible de les lancer avec la commande \texttt{pytest} à la racine du projet ou dans le dossier othello/. \\
    L’automatisation des tests est intégrée à la pipeline CI gitlab.                                                                                                                \\
    Ces tests couvrent au minimum 80\% du code écrit.                                                                                                                               \\
    \hline
\end{tabularx}

\vspace{1cm}
\noindent
\setlength{\arrayrulewidth}{1.5pt}
\renewcommand{\arraystretch}{1.5}
\arrayrulecolor{RoyalBlue}
\begin{tabularx}{\textwidth}{|X|}
    \hline
    \textbf{F.7 - Bugs}                                                                                                                                                                                                                 \\
    \hline
    Besoin non fonctionnel                                                                                                                                                                                                              \\
    \hline
    Terminé le DATE                                                                                                                                                                                                                     \\
    Par Matis, Rémy, Lucas, Gabriel                                                                                                                                                                                                     \\
    \hline
    L’implémentation des fonctionnalités du logiciel seront robustes - dans la mesure du possible - aux éventuelles mauvaises entrées de l’utilisateur, ou à d’autres cas amenant à des bugs.                                           \\
    \arrayrulecolor{MediumAquamarine}\hline
    \arrayrulecolor{RoyalBlue}
    Afin de pallier d'éventuels bugs menant à un crash du logiciel, les fonctionnalités sont testées le plus possible, et les erreurs d’entrées ou d’arguments seront traitées et renverront une erreur compréhensible à l’utilisateur. \\
    \hline
\end{tabularx}

\vspace{1cm}

\noindent
\setlength{\arrayrulewidth}{1.5pt}
\renewcommand{\arraystretch}{1.5}
\arrayrulecolor{RoyalBlue}
\begin{tabularx}{\textwidth}{|X|}
    \hline
    \textbf{F.8 - Performances}                                                                                                                  \\
    \hline
    Besoin non fonctionnel                                                                                                                       \\
    \hline
    Terminé le 07/04                                                                                                                             \\
    Par Matis                                                                                                                                    \\
    \hline
    Ajout de l’option \texttt{--benchmark} pour mesurer les performances des IA avec un choix d’heuristique ainsi que de mode.                   \\
    \arrayrulecolor{MediumAquamarine}\hline
    \arrayrulecolor{RoyalBlue}
    Afin de faire des tests variés, nous avons changé la configuration de base pour y accueillir des options d’IA des deux cotés, noir et blanc. \\
    Placés dans le répertoire benchmark, il suffit de lancer \texttt{run\_experiments.py}.                                                       \\
    \hline
\end{tabularx}

\vspace{1cm}

\noindent
\setlength{\arrayrulewidth}{1.5pt}
\renewcommand{\arraystretch}{1.5}
\arrayrulecolor{RoyalBlue}
\begin{tabularx}{\textwidth}{|X|}
    \hline
    \textbf{F.9 - Build-system}                                                                                                                       \\
    \hline
    Besoin non fonctionnel                                                                                                                            \\
    \hline
    Terminé le 04/02                                                                                                                                  \\
    Par Rémy                                                                                                                                          \\
    \hline
    Le build-system est basé sur pyproject.toml et septuptools.                                                                                       \\
    Les métadonnées du projet sont définies (nom, version, auteurs…) dans le pyproject.toml, ainsi que les dépendances et scripts associés au projet. \\
    De plus, les stratégies de test et de coverage sont également définies dans le pyproject.toml.                                                    \\
    \hline
\end{tabularx}

\vspace{1cm}

\noindent
\setlength{\arrayrulewidth}{1.5pt}
\renewcommand{\arraystretch}{1.5}
\arrayrulecolor{RoyalBlue}
\begin{tabularx}{\textwidth}{|X|}
    \hline
    \textbf{F.10 - Framework de tests}                            \\
    \hline
    Besoin non fonctionnel                                        \\
    \hline
    Terminé le 04/02                                              \\
    Par Matis et Lucas                                            \\
    \hline
    Mise en place de la pipeline CI/CD dans git.
    Avant chaque commit, les tests seront lancés par la pipeline. \\
    Utilisation du framework de tests Pytest et de Coverage.      \\
    \hline
\end{tabularx}

\vspace{1cm}

\noindent
\setlength{\arrayrulewidth}{1.5pt}
\renewcommand{\arraystretch}{1.5}
\arrayrulecolor{CornflowerBlue}
\begin{tabularx}{\textwidth}{|X|}
    \hline
    \textbf{F.11 - Gestion des options}                                                                                                                                             \\
    \hline
    Besoin fonctionnel                                                                                                                                                              \\
    \hline
    Terminé le 18/02                                                                                                                                                                \\
    Par Gabriel                                                                                                                                                                     \\
    \hline
    Les options en lignes de commandes pour le lancement d’une partie sont gérées par le module argparse qui construit un objet contenant les arguments donnés par l’utilisateur.   \\
    Les arguments passés au programme prennent systématiquement la priorité sur les
    paramètres définis dans le fichier de configuration othello, qui lui-même aura la priorité
    sur les paramètres par défaut.                                                                                                                                                  \\

    Les fonctions principales implémentées sont les suivantes :                                                                                                                     \\
    \texttt{create\_parser() -> ArgumentParser}, qui initialise et configure le
    parseur d’arguments ;                                                                                                                                                           \\ \texttt{parse\_args() -> tuple[str, dict[str, Any]]},
    qui parse les arguments, et gère les potentielles erreurs d’entrée : si
    l’utilisateur a entré de mauvais arguments, ou une option non reconnue ; et                                                                                                     \\
    \texttt{parse\_error(parser: ArgumentParser, message: str) -> None}, qui
    affiche des messages d’erreurs dans la console, en utilisant la sortie d’erreur
    standard.                                                                                                                                                                       \\

    \arrayrulecolor{MediumAquamarine}\hline
    \arrayrulecolor{CornflowerBlue}
    Validation du parseur avec une configuration par défaut, avec un fichier de configuration, et avec les différentes options disponibles.                                         \\
    Validation de la gestion d’erreurs avec des configurations invalides : option non existante ou dont la valeur donnée n’est pas acceptable, incompatibilité des modes de jeu.    \\
    \arrayrulecolor{MediumAquamarine}\hline
    \arrayrulecolor{CornflowerBlue}
    Nécessite la définition d’un fichier de configuration (F.40), afin de connaître toutes les options que le parseur doit proposer à l’utilisateur pour le lancement d’une partie. \\
    Nécessaire pour lancer le jeu (F.15), ainsi que les différents modes de jeu et l’affichage de messages à l’utilisateur (besoins F.16 à F.24).                                   \\
    \hline
\end{tabularx}

\vspace{1cm}

\noindent
\setlength{\arrayrulewidth}{1.5pt}
\renewcommand{\arraystretch}{1.5}
\arrayrulecolor{RoyalBlue}
\begin{tabularx}{\textwidth}{|X|}
    \hline
    \textbf{F.12 - Bibliothèque graphique}                            \\
    \hline
    Besoin non fonctionnel                                            \\
    \hline
    Terminé le 27/02                                                  \\
    Par Lucas                                                         \\
    \hline
    Utilisation de la bibliothèque graphique PyGObject.               \\
    Utilisation de la version 4.0 de Gtk.                             \\
    \hline
    \arrayrulecolor{RoyalBlue}
    Nécessaire pour l’implémentation de l’interface graphique (F.26). \\
    \hline
\end{tabularx}

\vspace{1cm}

\noindent
\setlength{\arrayrulewidth}{1.5pt}
\renewcommand{\arraystretch}{1.5}
\arrayrulecolor{RoyalBlue}
\begin{tabularx}{\textwidth}{|X|}
    \hline
    \textbf{F.14 - Nom du programme}                                                                                                          \\
    \hline
    Besoin non fonctionnel                                                                                                                    \\
    \hline
    Terminé le 09/02                                                                                                                          \\
    Par Lucas                                                                                                                                 \\
    \hline
    Le logiciel pourra être lancé avec l’exécutable ‘othello’.                                                                                \\
    Taper ‘othello’ dans l’invite de commande permet de lancer une partie en mode normal, joueur contre joueur, sur un plateau de huit cases. \\
    Il sera nécessaire de lancer le programme depuis un environnement virtuel.                                                                \\
    \hline
\end{tabularx}

\vspace{1cm}

\noindent
\setlength{\arrayrulewidth}{1.5pt}
\renewcommand{\arraystretch}{1.5}
\arrayrulecolor{CornflowerBlue}
\begin{tabularx}{\textwidth}{|X|}
    \hline
    \textbf{F.15 - Usage général}                                                                                      \\
    \hline
    Besoin fonctionnel                                                                                                 \\
    \hline
    Terminé le 18/02                                                                                                   \\
    Par Matis et Lucas                                                                                                 \\
    \hline
    Lancement du programme principal pour jouer une partie d’Othello avec la commande suivante :                       \\
    \texttt{othello [OPTIONS] [FILENAME]}                                                                              \\
    Pour cela, nous aurons besoin d’une initialisation de la partie.                                                   \\
    D’abord, récupérer les options données par l’utilisateur, et en récupérer une configuration pour la partie.        \\
    Récupérer une partie précédente depuis un fichier si précisé.                                                      \\
    Suivant le mode de jeu, lancer la boucle de jeu correspondante.                                                    \\
    Et afficher le jeu en interface graphique ou en lignes de commandes.                                               \\
    \hline
    Nécessite de lancer le programme avec le nom de l’exécutable (F.14).                                               \\
    Nécessite l’implémentation du parser (F.11), et des boucles de jeu pour un mode normal, blitz (F.21) et IA (F.24). \\
    Ainsi que l’implémentation de la CLI (F.25) et de la GUI (F.26).                                                   \\
    \hline
\end{tabularx}

\vspace{1cm}

\noindent
\setlength{\arrayrulewidth}{1.5pt}
\renewcommand{\arraystretch}{1.5}
\arrayrulecolor{CornflowerBlue}
\begin{tabularx}{\textwidth}{|X|}
    \hline
    \textbf{F.16 - Aide en ligne de commande}                                                                                                  \\
    \hline
    Besoin fonctionnel                                                                                                                         \\
    \hline
    Terminé le 11/02 et 10/03                                                                                                                  \\
    Par Gabriel, et Matis et Lucas                                                                                                             \\
    \hline
    Affichage de l’aide pour lancer le programme avec les options \texttt{-h} ou \texttt{--help}.                                              \\
    Construction de l’aide du lancement du programme avec le parseur d’argument directement : utilisation de l’argument “help=’'” de argparse. \\

    Affichage de l’aide et des règles du jeu durant la partie avec les options
    \texttt{?} et \texttt{r} respectivement.                                                                                                   \\ Lorsque l’utilisateur entre une
    commande non reconnue en cours de partie, l’aide et des exemples de commandes
    sont affichés à la suite de l’entrée de l’utilisateur.                                                                                     \\
    \arrayrulecolor{MediumAquamarine}\hline \arrayrulecolor{CornflowerBlue}
    Validation en vérifiant que les deux options pour afficher l’aide en dehors
    d’une partie affichent bien le message d’aide pour chaque option disponible.                                                               \\
    Vérification que ces deux options font terminer le programme avec
    \texttt{EXIT\_SUCCESS}.                                                                                                                    \\ Validation en vérifiant que les messages d’aide et
    de règles sont bien affichés dans leur intégralité lors d’un appel à ces
    commandes pendant une partie.\ Vérification qu’après l’affichage de ces
    messages, la partie continue, en attendant toujours une commande de la part du
    même joueur.                                                                                                                               \\ \arrayrulecolor{MediumAquamarine}\hline
    \arrayrulecolor{CornflowerBlue} Nécessite une implémentation du parseur
    d’arguments incluant des messages d’aide personnalisés (F.11).                                                                             \\ Nécessaire
    pour avoir une interface en ligne de commande complète (F.25).                                                                             \\ \hline
\end{tabularx}

\vspace{1cm}

\noindent
\setlength{\arrayrulewidth}{1.5pt}
\renewcommand{\arraystretch}{1.5}
\arrayrulecolor{CornflowerBlue}
\begin{tabularx}{\textwidth}{|X|}
    \hline
    \textbf{F.17 - Version}                                                                          \\
    \hline
    Besoin fonctionnel                                                                               \\
    \hline
    Terminé le 11/02                                                                                 \\
    Par Gabriel                                                                                      \\
    \hline
    Affichage de la version actuelle du logiciel avec les options \texttt{-v} ou \texttt{--version}. \\
    \arrayrulecolor{MediumAquamarine}\hline
    \arrayrulecolor{CornflowerBlue}
    Validation en vérifiant que les deux options permettent bien d’afficher la version correcte.
    Vérification que ces deux options font terminer le programme avec \texttt{EXIT\_SUCCESS}.        \\
    \arrayrulecolor{MediumAquamarine}\hline
    \arrayrulecolor{CornflowerBlue}
    Nécessite la définition de la version courante dans le fichier d’initialisation du logiciel.     \\
    \hline
\end{tabularx}

\vspace{1cm}

\noindent
\setlength{\arrayrulewidth}{1.5pt}
\renewcommand{\arraystretch}{1.5}
\arrayrulecolor{CornflowerBlue}
\begin{tabularx}{\textwidth}{|X|}
    \hline
    \textbf{F.19 - Mode debug}                                                                                                                                                                                                                \\
    \hline
    Besoin fonctionnel                                                                                                                                                                                                                        \\
    \hline
    Terminé le 27/02                                                                                                                                                                                                                          \\
    Par Gabriel                                                                                                                                                                                                                               \\
    \hline
    Activation du mode Debug avec les options \texttt{-d} ou \texttt{--debug} lors du lancement du programme.                                                                                                                                 \\
    Au cours de la partie, des messages de debug sont écrits dans un fichier “othello.log”.                                                                                                                                                   \\
    Ces messages préviennent, entre autres, de l’état du jeu courant, des fonctions appelées, des entrées de l’utilisateur et de leur traitement, ainsi que des potentielles erreurs rencontrées, avec un contexte et un traçage de l’erreur. \\

    Utilisation du module logging pour construire le logger, puis écriture de
    messages de debug pour pouvoir suivre les appels de fonctions et les erreurs en
    cours de partie.                                                                                                                                                                                                                          \\ Les fonctions principales implémentées sont les suivantes :
    \\ \texttt{logging\_config() -> None} qui configure le logger, en précisant le
    format d’affichage des messages, ainsi que le nom du fichier dans lequel écrire
    les messages de debug.                                                                                                                                                                                                                    \\ \texttt{log\_error\_message() -> None} qui affiche
    des messages d’erreurs customisés, avec un contexte affiché s'il est fourni, et
    un traçage de l’erreur relevée.                                                                                                                                                                                                           \\ \arrayrulecolor{MediumAquamarine}\hline
    \arrayrulecolor{CornflowerBlue} Validation en vérifiant que le logger est bien
    nommé “Othello”, et que l’on utilise toujours la même instance du logger pour
    tous les messages.                                                                                                                                                                                                                        \\ Vérification que si le mode Debug n’est pas activé par
    l’utilisateur, le logger ne s’initialise pas.                                                                                                                                                                                             \\ Validation en vérifiant que
    tous types de messages : informations de debug, erreurs avec et sans contexte
    soient bien écrits dans le fichier .log.                                                                                                                                                                                                  \\
    \arrayrulecolor{MediumAquamarine}\hline \arrayrulecolor{CornflowerBlue}
    Nécessaire pour avoir un affichage de messages pertinents pour un utilisateur
    versé dans la programmation (F.31).                                                                                                                                                                                                       \\ \hline
\end{tabularx}

\vspace{1cm}

\noindent
\setlength{\arrayrulewidth}{1.5pt}
\renewcommand{\arraystretch}{1.5}
\arrayrulecolor{CornflowerBlue}
\begin{tabularx}{\textwidth}{|X|}
    \hline
    \textbf{F.20 - Taille du plateau}                                                                                                                                                                    \\
    \hline
    Besoin fonctionnel                                                                                                                                                                                   \\
    \hline
    Terminé le 12/02                                                                                                                                                                                     \\
    Par Lucas                                                                                                                                                                                            \\
    \hline
    Possibilité de jouer sur un Othellier de 6×6, 8×8, 10×10, ou 12×12 cases.                                                                                                                            \\
    Représentations des différentes tailles de plateau avec les bitboards.                                                                                                                               \\
    Possibilité de passer d’un entier à une taille de plateau \texttt{BoardSize} et vice-versa sous couvert d’une taille valide.                                                                         \\
    \arrayrulecolor{MediumAquamarine}\hline
    \arrayrulecolor{CornflowerBlue}
    Validation en utilisant une énumération \texttt{BoardSize} contraignant les différentes tailles possibles pour un board et en ne passant les valeurs numériques qu’à la configuration des bitboards. \\
    Vérification de l’impossibilité de travailler avec des \texttt{Boardsize} de taille incohérente.                                                                                                     \\
    \arrayrulecolor{MediumAquamarine}\hline
    \arrayrulecolor{CornflowerBlue}
    Nécessite le début de l’implémentation des bitboards (F.42).                                                                                                                                         \\
    Nécessaire pour pouvoir jouer une partie : le plateau de base étant en 8 par 8 cases (F.15).                                                                                                         \\
    Nécessaire pour pouvoir configurer un plateau de jeu (F.41).                                                                                                                                         \\
    \hline
\end{tabularx}

\vspace{1cm}

\noindent
\setlength{\arrayrulewidth}{1.5pt}
\renewcommand{\arraystretch}{1.5}
\arrayrulecolor{CornflowerBlue}
\begin{tabularx}{\textwidth}{|X|}
    \hline
    \textbf{F.21 - Mode blitz}                                                                                                                                         \\
    \hline
    Besoin fonctionnel                                                                                                                                                 \\
    \hline
    Terminé le 20/02                                                                                                                                                   \\
    Par Matis                                                                                                                                                          \\
    \hline
    Les options \texttt{-b} ou \texttt{--blitz} permettent d’activer le mode blitz.                                                                                    \\
    Mise en place de la boucle de jeu pour le mode blitz : changement des joueurs, gestion du temps au cours de la partie, et passage en texte des valeurs des timers. \\

    Les fonctions principales implémentées sont les suivantes :                                                                                                        \\
    \texttt{change\_player(player: str) -> None} qui est utilisée pour mettre en
    oeuvre la rotation des joueurs.                                                                                                                                    \\ \texttt{is\_time\_up(player: str) -> bool}
    qui renvoie l’état d’un timer, s’il reste encore du temps ou qu’il s’est
    totalement écoulé.                                                                                                                                                 \\ \arrayrulecolor{MediumAquamarine}\hline
    \arrayrulecolor{CornflowerBlue} Validation en vérifiant que le changement de
    joueur pause le timer du joueur courant, démarre le timer du nouveau joueur, et
    change le joueur courant pour le nouveau joueur.                                                                                                                   \\ Vérification que lorsque
    l’un des timer arrive à zéro, la fonction \texttt{is\_time\_up} renvoie Vrai,
    et Faux sinon.                                                                                                                                                     \\ \arrayrulecolor{MediumAquamarine}\hline
    \arrayrulecolor{CornflowerBlue} Nécessite l’implémentation des timers pour le
    mode blitz (F.22).                                                                                                                                                 \\ Nécessaire pour l’implémentation d’un jeu fonctionnel
    plus complet (F.15).                                                                                                                                               \\ Nécessaire pour l’affichage du plateau (en CLI et GUI)
    (F.25, F.26, et F.27), ainsi que pour déterminer une fin de partie par faute de
    temps (F.36).                                                                                                                                                      \\ \hline
\end{tabularx}

\vspace{1cm}

\noindent
\setlength{\arrayrulewidth}{1.5pt}
\renewcommand{\arraystretch}{1.5}
\arrayrulecolor{CornflowerBlue}
\begin{tabularx}{\textwidth}{|X|}
    \hline
    \textbf{F.22 - Durée du blitz}                                                                                                                                                                                                        \\
    \hline
    Besoin fonctionnel                                                                                                                                                                                                                    \\
    \hline
    Terminé le 12/02                                                                                                                                                                                                                      \\
    Par Gabriel                                                                                                                                                                                                                           \\
    \hline
    Implémentation d’un timer pour le mode blitz.                                                                                                                                                                                         \\
    Ajouter un temps \texttt{-t TIME} ou \texttt{--time TIME} permet d’imposer un temps en minutes, choisi par l’utilisateur, limitant la durée de la partie en mode blitz. Par défaut, le temps limite est fixé à 30 minutes par joueur. \\
    Si le mode blitz n’est pas actif, cette option n’est pas considérée.                                                                                                                                                                  \\

    Les fonctions principales implémentées sont les suivantes:                                                                                                                                                                            \\
    \texttt{init(time\_limit: int) -> None} qui initialise les timers des deux
    joueurs à la valeur de time\_limit.                                                                                                                                                                                                   \\ \texttt{start\_timer(player: str) ->
        None} qui démarre le timer du joueur spécifié, et indique que le joueur
    spécifié est le joueur courant.                                                                                                                                                                                                       \\ \texttt{pause\_timer() -> None} qui stoppe
    le timer du joueur courant et update son temps restant.                                                                                                                                                                               \\
    \texttt{get\_remaining\_time(player: str) -> float} qui actualise le temps du
    joueur spécifié et le renvoie en secondes.                                                                                                                                                                                            \\ Les timers ne sont pas
    initialisés s’ils existent déjà. Il n’est pas possible de mettre un timer en
    pause s’il n’est pas en train de tourner dans un premier temps. Et le temps ne
    peut pas prendre de valeur négative, il est bloqué à zéro au minimum.                                                                                                                                                                 \\
    \arrayrulecolor{MediumAquamarine}\hline \arrayrulecolor{CornflowerBlue}
    Vérification des potentielles erreurs d’initialisation et de timer négatif.
    Validation en vérifiant que les timers sont bien initialisés avec la valeur de
    \texttt{time\_limit} donnée, et que le joueur courant est fixé au joueur aux
    pions noirs.                                                                                                                                                                                                                          \\ Vérification que lorsque l’on actualise le timer, si la valeur
    est négative, on fixe le temps à zéro.                                                                                                                                                                                                \\
    \arrayrulecolor{MediumAquamarine}\hline \arrayrulecolor{CornflowerBlue}
    Nécessaire pour l’implémentation du mode Blitz (F.21).                                                                                                                                                                                \\ \hline
\end{tabularx}

\vspace{1cm}

\noindent
\setlength{\arrayrulewidth}{1.5pt}
\renewcommand{\arraystretch}{1.5}
\arrayrulecolor{CornflowerBlue}
\begin{tabularx}{\textwidth}{|X|}
    \hline
    \textbf{F.24 - Mode IA}                                                                                                                                                                                                                                    \\
    \hline
    Besoin fonctionnel                                                                                                                                                                                                                                         \\
    \hline
    Terminé le 25/03                                                                                                                                                                                                                                           \\
    Par Rémy et Matis                                                                                                                                                                                                                                          \\
    \hline
    Les options \texttt{-a [COLOR]} et \texttt{--ai [COLOR]} permettent de lancer une partie en mode IA : l’utilisateur contre une intelligence artificielle, avec possibilité de spécifier la couleur du joueur IA, qui est de noir par défaut.               \\
    Un fichier a été créé pour les fonctionnalités IA.                                                                                                                                                                                                         \\
    \arrayrulecolor{MediumAquamarine}\hline
    \arrayrulecolor{CornflowerBlue}
    Vérification des heuristiques avec des profondeurs 1,2 puis différentes tailles de board.                                                                                                                                                                  \\
    \arrayrulecolor{MediumAquamarine}\hline
    \arrayrulecolor{CornflowerBlue}
    Nécessite toutes les fonctionnalités de l’IA : les différentes heuristiques implémentées (F.51, F.57), les algorithmes de recherche du meilleur coup : MiniMax (F.52) et Alpha-Beta Pruning (F.53), ainsi que la limite de profondeur de recherche (F.56). \\
    Nécessaire pour l’implémentation d’un jeu fonctionnel plus complet (F.15).                                                                                                                                                                                 \\
    \hline
\end{tabularx}

\vspace{1cm}

\noindent
\setlength{\arrayrulewidth}{1.5pt}
\renewcommand{\arraystretch}{1.5}
\arrayrulecolor{CornflowerBlue}
\begin{tabularx}{\textwidth}{|X|}
    \hline
    \textbf{F.25 - Interface en ligne de commande}                                                                                                                                                                                              \\
    \hline
    Besoin fonctionnel                                                                                                                                                                                                                          \\
    \hline
    Terminé le 20/02                                                                                                                                                                                                                            \\
    Par Matis                                                                                                                                                                                                                                   \\
    \hline
    Implémentation de l’interface en lignes de commandes avec un affichage clair des informations sur la partie en cours.                                                                                                                       \\
    Affichage de la configuration de la partie à son lancement.                                                                                                                                                                                 \\
    Affichage du numéro du tour courant et de l'historique des cinq derniers coups joués.                                                                                                                                                       \\
    Affichage du plateau de jeu courant, avec le nom des colonnes et lignes\,; les pions blancs et noirs représentés par les caractères `O' et `X' respectivement\,; et les coups possibles représentés par le caractère `\textperiodcentered'. \\
    Affichage du nom du joueur courant et des coups possibles en format \texttt{[colonne][ligne]} avec la ligne puis la colonne de la case concernée.                                                                                           \\
    Si la partie est en mode blitz, le temps restant de chaque joueur est affiché au format \texttt{[min:sec]}.                                                                                                                                 \\
    Et affichage d’une invite pour que l’utilisateur puisse entrer les commandes.                                                                                                                                                               \\
    La fonctions principale implémentée est la suivante :                                                                                                                                                                                       \\
    \texttt{get\_remaining\_time(player) -> None} Nous donne le temps restant d’un joueur pour ensuite l’afficher sur la CLI.                                                                                                                   \\

    Pour bien gérer les entrées de l’utilisateur, nous avons besoin d’un parseur de
    commandes, qui définit quelles chaînes de caractères sont acceptées, et quelles
    commandes elles permettent d’utiliser.                                                                                                                                                                                                      \\ Les fonctions principales
    implémentées sont les suivantes :                                                                                                                                                                                                           \\ \texttt{<print\_help() -> None} et
    \texttt{print\_rules() -> None} qui affichent l’aide en cours de partie, ainsi
    que des exemples de commandes, et les règles du jeu (en anglais).                                                                                                                                                                           \\
    \texttt{parse\_str(command\_str: str) -> CommandType} qui permet de prendre
    l’entrée de l'utilisateur, et de renvoyer une erreur et d’afficher l’aide si la
    commande n’est pas reconnue, ou de renvoyer le type de commande appelé.                                                                                                                                                                     \\
    \hline Nécessite l’implémentation du plateau, de la logique de jeu et de l’état
    courant (F.41 et F.43).                                                                                                                                                                                                                     \\ Nécessite les sous-besoins liés à l’affichage d’une
    partie : l’affichage du plateau, des coups joués et des coups possibles, de
    l’historique (besoins F.27, F.28, F.29), ainsi que de l'aide en cours de partie
    (F.16).                                                                                                                                                                                                                                     \\ Nécessite l'implémentation des besoins liés aux commandes de jeu :
    abandon de la partie, sauvegarde d’une partie et fermeture du programme,
    sauvegarde de l’historique, et recommencer la partie (besoins F.30, F.32, F.33,
    F.35).                                                                                                                                                                                                                                      \\ Nécessite l’implémentation des modes de jeu blitz (F.21) et IA
    (F.24).                                                                                                                                                                                                                                     \\ Nécessaire pour une implémentation complète de notre logiciel
    (F.15).                                                                                                                                                                                                                                     \\ \hline
\end{tabularx}

\vspace{1cm}

\noindent
\setlength{\arrayrulewidth}{1.5pt}
\renewcommand{\arraystretch}{1.5}
\arrayrulecolor{CornflowerBlue}
\begin{tabularx}{\textwidth}{|X|}
    \hline
    \textbf{F.26 - Interface graphique}                                                                                                                                                                                                                                                                                                                                                                                                                                                                                                                                                                                         \\
    \hline
    Besoin fonctionnel                                                                                                                                                                                                                                                                                                                                                                                                                                                                                                                                                                                                          \\
    \hline
    Terminé le 12/03                                                                                                                                                                                                                                                                                                                                                                                                                                                                                                                                                                                                            \\
    Par Lucas                                                                                                                                                                                                                                                                                                                                                                                                                                                                                                                                                                                                                   \\
    \hline
    Les options \texttt{-g} ou \texttt{--gui} permettent de lancer l’affichage de la partie avec une interface graphique. Au début de la partie, la configuration courante s’affiche dans la console.                                                                                                                                                                                                                                                                                                                                                                                                                           \\
    Une fenêtre graphique s’ouvre, avec pour titre “Othello”, contenant en haut à gauche le temps restant mis à jour en temps réel de chaque joueur si la partie est en mode blitz. Le plateau de jeu est au centre, avec à droite le nom du joueur courant ainsi que l’historique des 15 derniers coups joués.                                                                                                                                                                                                                                                                                                                 \\
    Le nom du jeu est également affiché.                                                                                                                                                                                                                                                                                                                                                                                                                                                                                                                                                                                        \\
    Le plateau est quant à lui dessiné en utilisant cairo. Nous affichons les cases de l’othellier, les pions de chaque couleur déjà posés ainsi que les pions posables par le joueur courant en semi-transparence.                                                                                                                                                                                                                                                                                                                                                                                                             \\
    La GUI est développée en utilisant les portages de GObject/GTK sur python: `PyGObject`, en visant la version 4 de GTK.                                                                                                                                                                                                                                                                                                                                                                                                                                                                                                      \\
    En dessous du plateau, le compte de pièces pour chaque joueur est affiché, et des boutons correspondant aux différentes commandes (abandon, sauvegarde de la partie, sauvegarde de l’historique et redémarrage d’une partie) sont placés tout en bas à gauche. En cas d'événement de jeu (fin de partie, demande de confirmation, etc.) des fenêtres modales pourront apporter de l’information supplémentaire au joueur (eg. cause de la fin de partie). Des fenêtres modales de confirmation seront également affichées lorsque l’utilisateur tente de redémarrer la partie, l’abandonner ou la quitter après sauvegarde. \\
    De plus, des fenêtres modales de choix de fichiers permettent à l’utilisateur de choisir l’emplacement de ses sauvegardes de parties et d’historiques.                                                                                                                                                                                                                                                                                                                                                                                                                                                                      \\
    \arrayrulecolor{MediumAquamarine}\hline
    \arrayrulecolor{CornflowerBlue}
    Nous avons essayé de tester les fonctionnalités de notre GUI avec Gtk, sans succès. Les seules fonctions de GUI testées actuellement sont celles qui ne nécessitent pas d’ouverture de fenêtre graphique.                                                                                                                                                                                                                                                                                                                                                                                                                   \\
    \arrayrulecolor{MediumAquamarine}\hline
    \arrayrulecolor{CornflowerBlue}
    Nécessite l’implémentation du plateau, de la logique de jeu et de l’état courant (F.41 et F.43).                                                                                                                                                                                                                                                                                                                                                                                                                                                                                                                            \\
    Nécessite les sous-besoins liés à l’affichage d’une partie : l’affichage du plateau, des coups joués et des coups possibles, de l’historique (besoins F.27, F.28, F.29).                                                                                                                                                                                                                                                                                                                                                                                                                                                    \\
    Nécessite l'implémentation des besoins liés aux commandes de jeu : abandon de la partie, sauvegarde d’une partie et fermeture du programme, sauvegarde de l’historique, et recommencer la partie (besoins F.30, F.32, F.33, F.35).                                                                                                                                                                                                                                                                                                                                                                                          \\
    Nécessite l’implémentation des modes de jeu blitz (F.21) et IA (F.24).                                                                                                                                                                                                                                                                                                                                                                                                                                                                                                                                                      \\
    Nécessaire pour une implémentation complète de notre logiciel (F.15).                                                                                                                                                                                                                                                                                                                                                                                                                                                                                                                                                       \\
    \hline
\end{tabularx}

\vspace{1cm}

\noindent
\setlength{\arrayrulewidth}{1.5pt}
\renewcommand{\arraystretch}{1.5}
\arrayrulecolor{CornflowerBlue}
\begin{tabularx}{\textwidth}{|X|}
    \hline
    \textbf{F.27 - Affichage du plateau}                                                                                                                                               \\
    \hline
    Besoin fonctionnel                                                                                                                                                                 \\
    \hline
    Terminé le 20/02                                                                                                                                                                   \\
    Par Matis                                                                                                                                                                          \\
    \hline
    L'Othellier est un plateau carré de taille paire, variable dans notre implémentation. Les différentes tailles possibles sont 6 par 6, 8 par 8, 10 par 10 ou 12 par 12.             \\
    Le plateau est de couleur unie. Les colonnes sont nommées de a à f, h, j, ou l, et les lignes sont numérotées de 1 à 6, 8, 10, ou 12, suivant la taille.                           \\
    L'interface graphique et l'interface en lignes de commande auront des représentations différentes du plateau de jeu.                                                               \\
    Les joueurs et leurs pions sont représentés par des symboles ou des pions de leur couleur, `X' pour les noirs, et `O' pour les blancs.                                             \\
    Lors de la partie, les cases sur lesquelles il est possible de jouer un coup ont un affichage spécial~: soit avec un pion transparent, soit avec le symbole `\textperiodcentered'. \\
    Et les cases sur lesquelles il n'y a pas de pion sont représentées par des cases vides ou par le symbole `\_'.                                                                     \\
    \hline
    Dépendances                                                                                                                                                                        \\
    \hline
\end{tabularx}

\vspace{1cm}

\noindent
\setlength{\arrayrulewidth}{1.5pt}
\renewcommand{\arraystretch}{1.5}
\arrayrulecolor{RoyalBlue}
\begin{tabularx}{\textwidth}{|X|}
    \hline
    \textbf{F.28 - Notation des coups}                                                                                                                                                                                       \\
    \hline
    Besoin non fonctionnel                                                                                                                                                                                                   \\
    \hline
    Terminé le 20/02                                                                                                                                                                                                         \\
    Par Matis                                                                                                                                                                                                                \\
    \hline
    Les coups joués suivent la notation au format [colonne][ligne] – plus clairement : [a-h][1-8] pour un plateau de taille 8×8.                                                                                             \\
    Ces notations sont suivies sur l’historique, et lorsque le joueur veut entrer un coup à jouer.                                                                                                                           \\
    Un coup n’est pas possible s’il est en dehors de la grille, qu’il ne permet pas de capturer un ou des pions adverses, ou que la case sélectionnée contient déjà un pion.                                                 \\
    Ou encore s’il est mal noté : colonne et ligne inversées, qu’il y a un espace entre la colonne et la ligne, ou plusieurs caractères pour l’un des deux.                                                                  \\
    En cas de mauvaise entrée, le programme indique au joueur qu’il s’est trompé, lui affiche l’aide en cours de partie, ainsi que des exemples de commandes qu’il peut entrer. Puis lui invite à entrer à nouveau son coup. \\
    \hline
    Dépendances                                                                                                                                                                                                              \\
    \hline
\end{tabularx}

\vspace{1cm}

\noindent
\setlength{\arrayrulewidth}{1.5pt}
\renewcommand{\arraystretch}{1.5}
\arrayrulecolor{CornflowerBlue}
\begin{tabularx}{\textwidth}{|X|}
    \hline
    \textbf{F.29 - Représentation de l’historique}                                                                                                                                                                                                                                                                                                                                                                              \\
    \hline
    Besoin fonctionnel                                                                                                                                                                                                                                                                                                                                                                                                          \\
    \hline
    Terminé le 25/02                                                                                                                                                                                                                                                                                                                                                                                                            \\
    Par Lucas                                                                                                                                                                                                                                                                                                                                                                                                                   \\
    \hline
    L'historique est représenté en notant pour chaque tour les coups des deux joueurs. Si un joueur n’a pas joué son tour, alors il n’est pas présent dans l’historique (comprendre qu’il n’y aura pas un \og X e3 O \_ \fg). Si un joueur a sauté un coup, car il ne pouvait pas jouer, alors \texttt{-1-1} est utilisé pour représenter le coup. Les commentaires sur une ligne sont acceptés et débutent par un \texttt{\#}. \\
    L'historique est affiché dans les deux interfaces proposées, et affiche les derniers coups joués en suivant le format de notation, ainsi que le joueur associé.                                                                                                                                                                                                                                                             \\
    Dans l’interface en ligne de commande, cinq coups sont affichés. Et dans l’interface graphique, les quinze derniers coups sont affichés.                                                                                                                                                                                                                                                                                    \\
    \arrayrulecolor{MediumAquamarine}\hline
    \arrayrulecolor{CornflowerBlue}
    Vérification qu’un historique issu d’une partie ramène bien à cette même partie.                                                                                                                                                                                                                                                                                                                                            \\
    Vérification qu’un historique incohérent ne résulte pas en une partie valide. Cependant, on peut avoir une partie sans historique, il sera possible de reprendre la partie, mais elle n’aura pas d’historique ancien.                                                                                                                                                                                                       \\
    \arrayrulecolor{MediumAquamarine}\hline
    \arrayrulecolor{CornflowerBlue}
    Nécessite un plateau de jeu, car intrinsèquement lié à ce dernier (F.41).                                                                                                                                                                                                                                                                                                                                                   \\
    Nécessaire pour le format de fichier (F.38).                                                                                                                                                                                                                                                                                                                                                                                \\
    \hline
\end{tabularx}

\vspace{1cm}

\noindent
\setlength{\arrayrulewidth}{1.5pt}
\renewcommand{\arraystretch}{1.5}
\arrayrulecolor{CornflowerBlue}
\begin{tabularx}{\textwidth}{|X|}
    \hline
    \textbf{F.30 - Abandon}                                                                                                                                                                                                                                        \\
    \hline
    Besoin fonctionnel                                                                                                                                                                                                                                             \\
    \hline
    Terminé le 12/02                                                                                                                                                                                                                                               \\
    Par Lucas                                                                                                                                                                                                                                                      \\
    \hline
    En cours de partie, le joueur a accès à une commande pour abandonner la partie en entrant \texttt{ff} dans l’interface en lignes de commande. Dans l’interface graphique, le joueur devra valider son choix après avoir appuyé sur le bouton \texttt{forfeit}. \\
    Un abandon de partie par un joueur entraîne la victoire de son adversaire, et termine la partie.                                                                                                                                                               \\
    \arrayrulecolor{MediumAquamarine}\hline
    \arrayrulecolor{CornflowerBlue}
    Vérification que l’appel de \texttt{forfeit} termine la partie.                                                                                                                                                                                                \\
    \arrayrulecolor{MediumAquamarine}\hline
    \arrayrulecolor{CornflowerBlue}
    Nécessite l'implémentation du plateau de jeu et de la logique de jeu (F.41 et F.43)                                                                                                                                                                            \\
    \hline
\end{tabularx}

\vspace{1cm}

\noindent
\setlength{\arrayrulewidth}{1.5pt}
\renewcommand{\arraystretch}{1.5}
\arrayrulecolor{CornflowerBlue}
\begin{tabularx}{\textwidth}{|X|}
    \hline
    \textbf{F.31 - Affichage des messages}                                                                                                                                                                                                                                                                                                                                                                                 \\
    \hline
    Besoin fonctionnel                                                                                                                                                                                                                                                                                                                                                                                                     \\
    \hline
    Terminé les 20/02 et 27/02                                                                                                                                                                                                                                                                                                                                                                                             \\
    Par Matis et Gabriel                                                                                                                                                                                                                                                                                                                                                                                                   \\
    \hline
    Des messages d’information seront affichés au joueur.                                                                                                                                                                                                                                                                                                                                                                  \\
    Dans l’interface en ligne de commande, le joueur est informé du tour, ainsi que du joueur courant, des cinq derniers coups joués. Le joueur peut entrer des commandes qui lui permettent de connaitre les règles du jeu, et l’aide du logiciel en cours de partie. Si la commande ou le coup entré est invalide, l’aide s’affiche automatiquement avec des exemples d’entrées pour utiliser les commandes disponibles. \\
    Dans l’interface graphique, le temps restant des deux joueurs est affiché, ainsi que l’historique des coups joués. Une validation par le joueur est attendue s'il décide d’abandonner la partie.                                                                                                                                                                                                                       \\
    Avec le mode debug d’activé, peu importe l’interface utilisée, un fichier ‘othello.log’ est créé au début de partie et contiendra des messages destinés au debug du programme. Ces messages seront soit d’ordre informatif, tout au long de la partie, soit des messages d’erreurs.                                                                                                                                    \\
    \hline
    Nécessite l’implémentation du mode debug (F.19).                                                                                                                                                                                                                                                                                                                                                                       \\
    Nécessaire pour l’implémentation complète de la CLI et de la GUI (F.25, F.26).                                                                                                                                                                                                                                                                                                                                         \\
    \hline
\end{tabularx}

\vspace{1cm}

\noindent
\setlength{\arrayrulewidth}{1.5pt}
\renewcommand{\arraystretch}{1.5}
\arrayrulecolor{CornflowerBlue}
\begin{tabularx}{\textwidth}{|X|}
    \hline
    \textbf{F.32 - Quitter et sauvegarder une partie}                                                                                                                                 \\
    \hline
    Besoin fonctionnel                                                                                                                                                                \\
    \hline
    Terminé le 12/02                                                                                                                                                                  \\
    Par Lucas                                                                                                                                                                         \\
    \hline
    À tout moment dans la partie, le joueur peut quitter la partie et la sauvegarder.                                                                                                 \\
    Dans l’interface graphique, le joueur peut cliquer sur des boutons associés à quitter la partie, et quitter et sauvegarder la partie.                                             \\
    Dans l’interface en lignes de commande, pour uniquement quitter la partie, le joueur peut entrer \texttt{q}, et pour quitter et sauvegarder la partie, il peut entrer \texttt{s}. \\
    Quitter la partie ferme simplement le programme.                                                                                                                                  \\
    Quitter et sauvegarder la partie enregistre la partie dans un fichier \texttt{name.sav}, avec le nom choisi par l’utilisateur, puis ferme le programme.                           \\
    \arrayrulecolor{MediumAquamarine}\hline
    \arrayrulecolor{CornflowerBlue}
    Stratégie de validation                                                                                                                                                           \\
    \arrayrulecolor{MediumAquamarine}\hline
    \arrayrulecolor{CornflowerBlue}
    Nécessaire pour l’implémentation complète de la CLI et de la GUI (F.25, F.26).                                                                                                    \\
    \hline
\end{tabularx}

\vspace{1cm}

\noindent
\setlength{\arrayrulewidth}{1.5pt}
\renewcommand{\arraystretch}{1.5}
\arrayrulecolor{CornflowerBlue}
\begin{tabularx}{\textwidth}{|X|}
    \hline
    \textbf{F.33 - Sauvegarde de l’historique}                                                                   \\
    \hline
    Besoin fonctionnel                                                                                           \\
    \hline
    Terminé le 12/02                                                                                             \\
    Par Lucas                                                                                                    \\
    \hline
    À tout moment dans la partie, le joueur peut sauvegarder l’historique des coups joués.                       \\
    Dans l’interface graphique, le joueur peut cliquer sur le bouton associé à l’enregistrement de l’historique. \\
    Dans l’interface en lignes de commande, pour sauvegarder l’historique, il peut entrer \texttt{sh}.           \\
    \arrayrulecolor{MediumAquamarine}\hline
    \arrayrulecolor{CornflowerBlue}
    Stratégie de validation                                                                                      \\
    \arrayrulecolor{MediumAquamarine}\hline
    \arrayrulecolor{CornflowerBlue}
    Nécessaire pour l’implémentation complète de la CLI et de la GUI (F.25, F.26).                               \\
    \hline
\end{tabularx}

\vspace{1cm}

\noindent
\setlength{\arrayrulewidth}{1.5pt}
\renewcommand{\arraystretch}{1.5}
\arrayrulecolor{CornflowerBlue}
\begin{tabularx}{\textwidth}{|X|}
    \hline
    \textbf{F.35 - Recommencer une partie}                                                                                                                             \\
    \hline
    Besoin fonctionnel                                                                                                                                                 \\
    \hline
    Terminé le 12/02                                                                                                                                                   \\
    Par Lucas                                                                                                                                                          \\
    \hline
    À tout moment dans la partie, le joueur peut recommencer une partie.                                                                                               \\
    La partie en cours est arrêtée, et une nouvelle partie débute. La configuration de la partie précédente est gardée, et le plateau de jeu reprend son état initial. \\
    Dans l’interface graphique, le joueur peut cliquer sur le bouton associé à recommencer la partie.                                                                  \\
    Dans l’interface en lignes de commande, pour recommencer la partie, il peut entrer \texttt{restart}.                                                               \\
    \arrayrulecolor{MediumAquamarine}\hline
    \arrayrulecolor{CornflowerBlue}
    Stratégie de validation                                                                                                                                            \\
    \arrayrulecolor{MediumAquamarine}\hline
    \arrayrulecolor{CornflowerBlue}
    Nécessaire pour l’implémentation complète de la CLI et de la GUI (F.25, F.26).                                                                                     \\
    \hline
\end{tabularx}

\vspace{1cm}

\noindent
\setlength{\arrayrulewidth}{1.5pt}
\renewcommand{\arraystretch}{1.5}
\arrayrulecolor{CornflowerBlue}
\begin{tabularx}{\textwidth}{|X|}
    \hline
    \textbf{F.36 - Fin de partie}                                                                                                                                            \\
    \hline
    Besoin fonctionnel                                                                                                                                                       \\
    \hline
    Terminé les 12/02 et 20/02                                                                                                                                               \\
    Par Lucas, et Matis                                                                                                                                                      \\
    \hline
    Une fin de partie est détectée si le plateau de jeu n’a plus aucune case vide, ou si aucun des deux joueurs ne peut poser de pion.                                       \\
    Si la partie est lancée en mode blitz, un timer atteignant zéro entraîne également une fin de la partie.                                                                 \\
    Le gagnant est le joueur ayant le plus de pions sur l’othellier. Dans le cas où les deux joueurs auraient le même nombre de points, la partie se termine en une égalité. \\
    \arrayrulecolor{MediumAquamarine}\hline
    \arrayrulecolor{CornflowerBlue}
    Stratégie de validation                                                                                                                                                  \\
    \arrayrulecolor{MediumAquamarine}\hline
    \arrayrulecolor{CornflowerBlue}
    Nécessite l’implémentation du mode blitz (F.21).                                                                                                                         \\
    Nécessaire pour l’implémentation complète de la CLI et de la GUI (F.25, F.26).                                                                                           \\
    \hline
\end{tabularx}

\vspace{1cm}

\noindent
\setlength{\arrayrulewidth}{1.5pt}
\renewcommand{\arraystretch}{1.5}
\arrayrulecolor{RoyalBlue}
\begin{tabularx}{\textwidth}{|X|}
    \hline
    \textbf{F.38 - Format de fichier simplifié}                                                                                                                                                    \\
    \hline
    Besoin non fonctionnel                                                                                                                                                                         \\
    \hline
    Terminé le 22/02                                                                                                                                                                               \\
    Par Lucas                                                                                                                                                                                      \\
    \hline
    Le format de fichier de sauvegarde de partie suit les normes fixées en début de projet~: fichier au format ASCII simplifié, et sauvegarde de la partie en fichier \texttt{.sav}.               \\
    Les fichiers peuvent être chargés afin de reprendre la partie depuis l’état décrit dans le fichier.                                                                                            \\
    Le fichier de sauvegarde d’une partie contient la couleur du joueur auquel c’est le tour de jouer, le plateau de jeu dans son dernier état, et l’historique des coups joués lors de la partie. \\
    En cas d’erreur (caractère inattendu ou fin de fichier avant d’avoir terminé la lecture), une erreur est remontée contenant la ligne ainsi qu’un message décrivant ce dernier.                 \\
    \arrayrulecolor{MediumAquamarine}\hline
    \arrayrulecolor{RoyalBlue}
    Stratégie de validation                                                                                                                                                                        \\
    \arrayrulecolor{MediumAquamarine}\hline
    \arrayrulecolor{RoyalBlue}
    Nécessaire pour le bon fonctionnement du jeu (F.15).                                                                                                                                           \\
    \hline
\end{tabularx}

\vspace{1cm}

\noindent
\setlength{\arrayrulewidth}{1.5pt}
\renewcommand{\arraystretch}{1.5}
\arrayrulecolor{RoyalBlue}
\begin{tabularx}{\textwidth}{|X|}
    \hline
    \textbf{F.39 - Format de l’historique}                                                                                                                                                                                                                                                   \\
    \hline
    Besoin non fonctionnel                                                                                                                                                                                                                                                                   \\
    \hline
    Terminé le 22/02                                                                                                                                                                                                                                                                         \\
    Par Lucas                                                                                                                                                                                                                                                                                \\
    \hline
    La sauvegarde de l’historique de la partie sera dans un fichier \texttt{.othellorc}.                                                                                                                                                                                                     \\
    Un tour comprend le coup d’un joueur ainsi que le coup de son adversaire.                                                                                                                                                                                                                \\
    Une ligne du fichier représentant l’historique se présente de façon suivante~: \texttt{[Numéro du tour]. X [coup joué] O [coup joué]}.                                                                                                                                                   \\
    \texttt{[coup joué]} vaut soit un coup valide (e.g. \texttt{a1}, \texttt{e3} \ldots) soit \texttt{-1-1} en cas de tour passé. Un coup pas encore joué dans un tour (e.g. noir a joué, mais pas blanc) sera représenté de la manière suivante~: \texttt{[Numéro du tour]. X [coup joué]}. \\
    En cas d’erreur (caractère inattendu ou fin de fichier avant d’avoir terminé la lecture), une erreur est remontée contenant la ligne ainsi qu’un message décrivant ce dernier.                                                                                                           \\
    \arrayrulecolor{MediumAquamarine}\hline
    \arrayrulecolor{RoyalBlue}
    Stratégie de validation                                                                                                                                                                                                                                                                  \\
    \arrayrulecolor{MediumAquamarine}\hline
    \arrayrulecolor{RoyalBlue}
    Nécessaire pour le bon fonctionnement du jeu (F.15).                                                                                                                                                                                                                                     \\
    \hline
\end{tabularx}

\vspace{1cm}

\noindent
\setlength{\arrayrulewidth}{1.5pt}
\renewcommand{\arraystretch}{1.5}
\arrayrulecolor{RoyalBlue}
\begin{tabularx}{\textwidth}{|X|}
    \hline
    \textbf{F.40 - Fichier de configuration}                                                                                                                                                                                                   \\
    \hline
    Besoin non fonctionnel                                                                                                                                                                                                                     \\
    \hline
    Terminé le 20/02                                                                                                                                                                                                                           \\
    Par Rémy et Gabriel                                                                                                                                                                                                                        \\
    \hline
    Définition du fichier de configuration, contenant les valeurs des options utilisées lors d’une partie.                                                                                                                                     \\
    Le fichier de configuration suit le format INI, et sera enregistré dans un fichier \texttt{.othellorc}.                                                                                                                                    \\
    Le fichier contient les valeurs des indications sur la taille du plateau, et le mode de jeu, ainsi que les options de blitz et d’IA.                                                                                                       \\
    Une configuration par défaut est définie, pour une partie avec un plateau en 8 par 8, en mode normal. Si des options sont précisées au lancement de la partie, elles remplaceront les valeurs par défaut dans le fichier de configuration. \\
    Un fichier de configuration erroné renverra une erreur et annulera le lancement du programme.                                                                                                                                              \\

    Les principales fonctions implémentées sont les suivantes~:                                                                                                                                                                                \\
    \texttt{save\_config(config: dict, filename\_prefix: str = "current\_config")
        -> None} qui sauvegarde la configuration donnée sous forme de dictionnaire,
    dans un fichier de sauvegarde.                                                                                                                                                                                                             \\ Et \texttt{load\_config(filename\_prefix: str
        = "current\_config") -> dict} qui charge une configuration depuis un fichier,
    dans un dictionnaire.                                                                                                                                                                                                                      \\ \arrayrulecolor{MediumAquamarine}\hline
    \arrayrulecolor{RoyalBlue} Validation en vérifiant que lancer le jeu sans
    aucune option charge bien la configuration par défaut.                                                                                                                                                                                     \\ Validation en
    vérifiant que le fichier de sauvegarde contient bien toutes les options et
    vérification que la configuration est modifiée lorsque l’on lance le jeu avec
    des options.                                                                                                                                                                                                                               \\ \arrayrulecolor{MediumAquamarine}\hline
    \arrayrulecolor{RoyalBlue} Nécessaire pour l’implémentation d’un jeu complet
    (F.15).                                                                                                                                                                                                                                    \\ Nécessite l’implémentation du plateau de jeu et de l’état du jeu
    (F.41, F.43).                                                                                                                                                                                                                              \\ \hline
\end{tabularx}

\vspace{1cm}

\noindent
\setlength{\arrayrulewidth}{1.5pt}
\renewcommand{\arraystretch}{1.5}
\arrayrulecolor{CornflowerBlue}
\begin{tabularx}{\textwidth}{|X|}
    \hline
    \textbf{F.41 - Plateau de jeu}                                                                                                                                                                                                                                                                                                                                                           \\
    \hline
    Besoin fonctionnel                                                                                                                                                                                                                                                                                                                                                                       \\
    \hline
    Terminé le 12/02                                                                                                                                                                                                                                                                                                                                                                         \\
    Par Lucas                                                                                                                                                                                                                                                                                                                                                                                \\
    \hline
    Le plateau de jeu, représenté par le module \texttt{othello\_board}, implémente une partie de la logique de jeu et sert également de modèle, gérant l’état du jeu ainsi que les changements d’état. Il peut être initialisé à partir d’une simple taille ou à partir d’une taille et de deux bitboards pour charger une partie. On précisera alors le joueur courant dans l’état du jeu. \\
    Il définit également \texttt{BoardSize} les tailles de plateau possibles dans le jeu, \texttt{Color} les couleurs possibles pour une case (noir, blanc ou vide), et un override de \texttt{\_\_invert\_\_} pour faciliter les changements de tours ainsi que quelques exceptions liées à la logique de jeu et aux énumérations définies.                                                 \\
    Il part d’une taille de l’énumération \texttt{BoardSize} pour générer des bitboards en conséquence.                                                                                                                                                                                                                                                                                      \\
    Il maintient deux bitboards à jour, un pour les pions noirs et un pour les pions blancs. Il garde également l’état du joueur courant.                                                                                                                                                                                                                                                    \\
    Il permet de vérifier s’il est dans un état de fin de partie grâce à \texttt{is\_game\_over() -> bool}, pour cela on regarde simplement qu’aucun des deux joueurs ne peut jouer un coup.                                                                                                                                                                                                 \\
    Il implémente les algorithmes sur les bitboards liés aux règles de jeu d’Othello à savoir :                                                                                                                                                                                                                                                                                              \\
    \texttt{line\_cap\_move(current\_player: Color)} -- Bitboard qui génère un bitboard de coups possibles à partir d’une configuration de jeu (bitboards + joueur courant) ;                                                                                                                                                                                                                \\
    \texttt{line\_cap(x\_coord: int, y\_coord: int, current\_player: Color)} -- Bitboard qui renvoie l’état du jeu en fonction de la capture d’un pion par le joueur courant.                                                                                                                                                                                                                \\
    Ces deux fonctions sont pures, elles renvoient un nouvel état sans altérer l’état courant.                                                                                                                                                                                                                                                                                               \\
    Pour jouer un coup (valide), on appelle \texttt{play(x\_coord: int, y\_coord: int)} qui -- si le coup est possible -- l’applique au joueur courant et met à jour l’état du plateau.                                                                                                                                                                                                      \\
    Bien que reposant initialement massivement sur les bitboards, étant donné la fréquence de l’appel de \texttt{line\_cap\_move}, son code est optimisé. Les appels successifs de \texttt{shift} ont été inlinés et les bitboards intermédiaires supprimés.                                                                                                                                 \\
    Le plateau de jeu maintient également l’état de l’historique de la partie, en gardant pour chaque coup joué les deux bitboards précédant le coup, le coup et le joueur.                                                                                                                                                                                                                  \\
    On peut réinitialiser l’état du plateau en appelant la méthode \texttt{reset()}.                                                                                                                                                                                                                                                                                                         \\
    Enfin, il offre les fonctions d’export d’un état de jeu vers des \texttt{strings}, permettant la sauvegarde de l’état du jeu : \texttt{export() -> str}, \texttt{export\_history() -> str}.                                                                                                                                                                                              \\
    Il fournit aussi des fonctions pratiques telles que :                                                                                                                                                                                                                                                                                                                                    \\
    \texttt{get\_turn\_id() -> int}, qui renvoie le numéro du tour courant ;                                                                                                                                                                                                                                                                                                                 \\
    \texttt{get\_last\_play() -> Optional[Move]}, qui renvoie le dernier coup joué si disponible ;                                                                                                                                                                                                                                                                                           \\
    \texttt{pop()}, qui renvoie le plateau à l’état n-1, utile pour les algorithmes d’IA.                                                                                                                                                                                                                                                                                                    \\                                                                                                                                                                                                                                                                                                                                   \\
    \hline
\end{tabularx}

\vspace{1cm}

\noindent
\setlength{\arrayrulewidth}{1.5pt}
\renewcommand{\arraystretch}{1.5}
\arrayrulecolor{CornflowerBlue}
\begin{tabularx}{\textwidth}{|X|}
    \arrayrulecolor{MediumAquamarine}\hline
    \arrayrulecolor{CornflowerBlue}
    On vérifie l’initialisation correcte d’un plateau de jeu en fonction de la taille passée au constructeur.                                                                                                                                                                   \\
    On vérifie que l’algorithme \texttt{line\_cap\_move} renvoie des données cohérentes en fonction de l’état du jeu et de son joueur courant, sur plusieurs états montrant différentes caractéristiques (position de départ, plateau concave/convexe, absence de coups, etc.). \\
    On vérifie également que, donnée une position de capture, \texttt{line\_cap} renvoie un état cohérent avec la capture voulue.                                                                                                                                               \\
    On vérifie qu’un export de partie ou d’historique correspond bien à la partie ou à l’historique correspondant.                                                                                                                                                              \\
    On vérifie que la détection de \texttt{game\_over} est bien fonctionnelle.                                                                                                                                                                                                  \\
    On vérifie qu’il n’est pas possible de jouer un coup impossible pour le joueur courant dans un état \texttt{S}.                                                                                                                                                             \\
    On vérifie le bon fonctionnement de \texttt{\_\_invert\_\_} sur \texttt{Color}.                                                                                                                                                                                             \\
    On vérifie le bon fonctionnement de \texttt{pop}, à savoir qu’il renvoie le dernier coup si et seulement si il est disponible.                                                                                                                                              \\
    On vérifie également que \texttt{restart} réinitialise bien l’état du \texttt{board}.                                                                                                                                                                                       \\
    \arrayrulecolor{MediumAquamarine}\hline
    \arrayrulecolor{CornflowerBlue}
    Nécessite l'implémentation du module Bitboard (F.42).                                                                                                                                                                                                                       \\
    \hline
\end{tabularx}

\vspace{1cm}

\noindent
\setlength{\arrayrulewidth}{1.5pt}
\renewcommand{\arraystretch}{1.5}
\arrayrulecolor{CornflowerBlue}
\begin{tabularx}{\textwidth}{|X|}
    \hline
    \textbf{F.42 - Module bitboard}                                                                                                                                                                                                                                                                                              \\
    \hline
    Besoin fonctionnel                                                                                                                                                                                                                                                                                                           \\
    \hline
    Terminé le 12/02                                                                                                                                                                                                                                                                                                             \\
    Par Lucas                                                                                                                                                                                                                                                                                                                    \\
    \hline
    Les \texttt{bitboards} sont utilisés pour représenter les pièces de chaque joueur, mais également par les algorithmes de calcul des coups possibles et de capture.                                                                                                                                                           \\
    Les \texttt{bitboards} utilisent les entiers de Python, ce qui permet notamment de travailler sur un nombre arbitraire de bits, moyennant l’utilisation des bons masques pour éviter tout dépassement et pallier les cas fréquents où les plateaux sont représentés avec un certain nombre de zéros côté bits de poids fort. \\
    Des masques sont également utilisés pour gérer les cas de dépassement lors des opérations de \texttt{shift}.                                                                                                                                                                                                                 \\
    La création d’un \texttt{Bitboard} mémorise certaines caractéristiques (par exemple les masques utilisés) afin d’éviter de les régénérer à chaque opération.                                                                                                                                                                 \\
    On implémente également un \texttt{popcount} efficace ainsi qu’une fonction pour récupérer les positions des bits à \texttt{1} sur un \texttt{Bitboard} : \texttt{popcount() -> int}, \texttt{get\_hot\_bit\_coordinates() -> list[tuple[int, int]]}.                                                                        \\
    \arrayrulecolor{MediumAquamarine}\hline
    \arrayrulecolor{CornflowerBlue}
    On vérifie l’initialisation correcte de \texttt{bitboards} de différentes tailles.                                                                                                                                                                                                                                           \\
    On vérifie que les accès à des indices hors limites renvoient une erreur (\texttt{IndexError}).                                                                                                                                                                                                                              \\
    On vérifie qu’on peut bien \texttt{set} et \texttt{get} des bits à des positions arbitraires (dans les limites du \texttt{bitboard}).                                                                                                                                                                                        \\
    On vérifie que les \texttt{shifts} sont correctement effectués.                                                                                                                                                                                                                                                              \\
    On vérifie que les valeurs renvoyées par \texttt{popcount} et \texttt{get\_hot\_bit\_coordinates} sont cohérentes.                                                                                                                                                                                                           \\
    On teste les différentes surcharges d’opérateurs.                                                                                                                                                                                                                                                                            \\
    On fait également quelques tests avec des valeurs pseudo-aléatoires.                                                                                                                                                                                                                                                         \\
    \hline
\end{tabularx}

\vspace{1cm}

\noindent
\setlength{\arrayrulewidth}{1.5pt}
\renewcommand{\arraystretch}{1.5}
\arrayrulecolor{CornflowerBlue}
\begin{tabularx}{\textwidth}{|X|}
    \hline
    \textbf{F.43 - État du jeu} \\
    \hline
    Besoin fonctionnel          \\
    \hline
    Terminé le 12/02            \\
    Par Lucas                   \\
    \hline
    Explication du              \\
    besoin                      \\
    \arrayrulecolor{MediumAquamarine}\hline
    \arrayrulecolor{CornflowerBlue}
    Stratégie de validation     \\
    \arrayrulecolor{MediumAquamarine}\hline
    \arrayrulecolor{CornflowerBlue}
    Dépendances                 \\
    \hline
\end{tabularx}

\vspace{1cm}

\noindent
\setlength{\arrayrulewidth}{1.5pt}
\renewcommand{\arraystretch}{1.5}
\arrayrulecolor{CornflowerBlue}
\begin{tabularx}{\textwidth}{|X|}
    \hline
    \textbf{F.44 - Fonctions de base}                                                                                                                                                                                                  \\
    \hline
    Besoin fonctionnel                                                                                                                                                                                                                 \\
    \hline
    Terminé le 27/02                                                                                                                                                                                                                   \\
    Par Lucas                                                                                                                                                                                                                          \\
    \hline
    L’interface graphique permet au joueur de jouer une partie avec autant de fonctionnalités disponibles que dans l’interface en lignes de commandes.                                                                                 \\
    Le lancement d’une partie avec options et en interface graphique se fait de la même manière que pour lancer le jeu en interface en lignes de commandes, en ajoutant l’option \texttt{<--gui>}.                                     \\
    Il est possible de lancer une partie à partir d’un fichier de sauvegarde.                                                                                                                                                          \\
    Le mode \texttt{blitz} est disponible, ainsi que le mode \texttt{IA}, avec toutes leurs options.                                                                                                                                   \\
    L’historique de la partie en cours est affiché, ainsi que le temps restant de chaque joueur en mode \texttt{blitz}.                                                                                                                \\
    Il est possible d’utiliser les commandes pour recommencer la partie, abandonner, quitter ou sauvegarder et quitter la partie.                                                                                                      \\
    \hline
    Nécessaire pour le fonctionnement de la GUI (F.26).                                                                                                                                                                                \\
    Nécessite le fonctionnement complet du jeu (F.15), plus précisément : des différents modes de jeu (besoins F.21, F.24), des commandes en cours de partie (F.30, F.32, F.33, F.35), et de l’affichage de la partie en cours (F.48). \\
    \hline
\end{tabularx}

\vspace{1cm}

\noindent
\setlength{\arrayrulewidth}{1.5pt}
\renewcommand{\arraystretch}{1.5}
\arrayrulecolor{RoyalBlue}
\begin{tabularx}{\textwidth}{|X|}
    \hline
    \textbf{F.48 - Affichage d’une partie en cours}                                                                                                                                                      \\
    \hline
    Besoin non fonctionnel                                                                                                                                                                               \\
    \hline
    Terminé le 27/02                                                                                                                                                                                     \\
    Par Lucas                                                                                                                                                                                            \\
    \hline
    L’interface graphique affiche le plateau de jeu, actualisé à chaque coup avec les nouveaux pions posés et capturés, ainsi que l’affichage des cases sur lesquelles il est possible de jouer un coup. \\
    L’affichage comprend également un historique sur le côté droit, avec les quinze derniers coups joués.                                                                                                \\
    \hline
    Nécessaire pour le fonctionnement de la GUI (F.26), ainsi que pour compléter les fonctions de bases dans la GUI (F.44).                                                                              \\
    Nécessite l’affichage du plateau, des coups et de l’historique (F.27, F.28, F.49).                                                                                                                   \\
    \hline
\end{tabularx}

\vspace{1cm}

\noindent
\setlength{\arrayrulewidth}{1.5pt}
\renewcommand{\arraystretch}{1.5}
\arrayrulecolor{RoyalBlue}
\begin{tabularx}{\textwidth}{|X|}
    \hline
    \textbf{F.49 - Affichage des coups joués}                                                    \\
    \hline
    Besoin non fonctionnel                                                                       \\
    \hline
    Terminé le 27/02                                                                             \\
    Par Lucas                                                                                    \\
    \hline
    Affichage de l’historique des quinze derniers coups joués dans la partie au format suivant : \\
    \texttt{[player] has placed a piece at [cell]}                                               \\
    L’historique sur la GUI sera affiché sur la droite du plateau.                               \\
    \hline
    Nécessaire pour l’affichage d’une partie en cours pour la GUI (F.48, et F.26 par extension). \\
    Nécessite l’implémentation de l’historique (F.29).                                           \\
    \hline
\end{tabularx}

\vspace{1cm}

\noindent
\setlength{\arrayrulewidth}{1.5pt}
\renewcommand{\arraystretch}{1.5}
\arrayrulecolor{CornflowerBlue}
\begin{tabularx}{\textwidth}{|X|}
    \hline
    \textbf{F.51 Heuristique d’évaluation de position} \\
    \hline
    Besoin fonctionnel \\
    \hline
    Terminé le 24/02 \\
    Par Rémy \\
    \hline
    Le programme contient plusieurs heuristiques capables de prendre un état du board en entrée et de renvoyer un score en fonction de différents critères.
    Les critères choisis pour évaluer un board sont les suivants : la capture des coins, le nombre de pierres de notre couleur, le nombre de coups possibles, ainsi que tous ces critères combinés.

    Les fonctions d’heuristiques utilisées sont les suivantes : \\

    \texttt{corners\_captured\_heuristic(board: OthelloBoard, max\_player: Color) -> int} \\
    \texttt{coin\_parity\_heuristic(board: OthelloBoard, max\_player: Color) -> int} \\
    \texttt{mobility\_heuristic(board: OthelloBoard, max\_player: Color) -> int} \\
    \texttt{all\_in\_one\_heuristic(board: OthelloBoard, max\_player: Color) -> int} \\

    qui renvoient toutes une évaluation en fonction des critères respectivement cités ci-dessus. \\
    \arrayrulecolor{MediumAquamarine}\hline
    \arrayrulecolor{CornflowerBlue}
    La stratégie de validation de ce besoin est assez triviale, il nous suffit de créer des plateaux de jeu prédéfinis où l’on connaît déjà le score que chaque fonction doit attribuer. Il suffit donc de comparer le vrai score avec celui renvoyé par chaque heuristique. \\
    \arrayrulecolor{MediumAquamarine}\hline
    \arrayrulecolor{CornflowerBlue}
    Nécessaire pour implémenter les algorithmes de recherche Minimax (F.52) et Alpha-Beta (F.53) \\
    \hline
\end{tabularx}

\vspace{1cm}

\noindent
\setlength{\arrayrulewidth}{1.5pt}
\renewcommand{\arraystretch}{1.5}
\arrayrulecolor{CornflowerBlue}
\begin{tabularx}{\textwidth}{|X|}
    \hline
    \textbf{F.52 Algorithme de recherche Minimax} \\
    \hline
    Besoin fonctionnel \\
    \hline
    Terminé le 24/02 \\
    Par Rémy \\
    \hline
    L’option \texttt{--ai-mode minimax} permet de spécifier l’algorithme de recherche utilisé par l’IA, et utiliser Minimax pour trouver son meilleur coup.
    Par défaut, le joueur IA utilise l’algorithme Minimax pour sa recherche.

    L’algorithme Minimax est l’algorithme d’exploration de l’arbre de jeu utilisé par défaut. Il prend l’état d’un plateau, la profondeur de l’arbre à explorer, le joueur IA (qui souhaite jouer le meilleur coup), et la fonction heuristique qui évaluera les feuilles de l’arbre : \\

    \texttt{minimax(board: OthelloBoard, depth: int, max\_player: Color, heuristic: Callable) -> float} \\
    \arrayrulecolor{MediumAquamarine}\hline
    \arrayrulecolor{CornflowerBlue}
    Pour vérifier le bon fonctionnement, on prend un plateau dont on connaît le meilleur coup à jouer, ainsi que le score de ce dernier en fonction de l’heuristique choisie. On peut ensuite comparer le résultat de Minimax en profondeur 1 avec le véritable score calculé au préalable.

    Pour vérifier les bugs à des profondeurs plus importantes, il est vite compliqué de prévoir quels sont les meilleurs coups à la main. On a donc fait jouer Minimax contre un joueur random (jouant des coups aléatoires) avec plusieurs profondeurs afin de s’assurer qu’aucun bug n’était caché. L’objectif était d’avoir un pourcentage élevé de victoire face à un joueur random. \\
    \arrayrulecolor{MediumAquamarine}\hline
    \arrayrulecolor{CornflowerBlue}
    L'implémentation des heuristiques (F.51) est nécessaire pour l'évaluation des plateau dans l'algorithme minimax.\\
    Minimax est nécessaire pour implémenter le mode IA (F.24) \\
    \hline
\end{tabularx}

\vspace{1cm}

\noindent
\setlength{\arrayrulewidth}{1.5pt}
\renewcommand{\arraystretch}{1.5}
\arrayrulecolor{CornflowerBlue}
\begin{tabularx}{\textwidth}{|X|}
    \hline
    \textbf{F.53 AlphaBeta pruning} \\
    \hline
    Besoin fonctionnel \\
    \hline
    Terminé le 18/03 \\
    Par Rémy \\
    \hline
    L’option \texttt{--ai-mode ab} permet de spécifier l’algorithme de recherche utilisé par l’IA, et utiliser l’AlphaBeta pruning, une amélioration de l’algorithme Minimax, pour trouver son meilleur coup.

    L’algorithme AlphaBeta est similaire à l’algorithme Minimax. Il renvoie donc le même résultat. La différence de cet algorithme est l’ajout de variables Alpha et Beta, permettant de garder en mémoire le meilleur coup ami et ennemi, ce qui permet de sauter l’exploration de certaines branches si l’on sait qu’elles ne permettront pas de trouver un meilleur coup.

    L’implémentation de la fonction est donc également similaire : \\

    \texttt{alphabeta(board: OthelloBoard, depth: int, alpha: int, beta: int, max\_player: Color, heuristic: Callable) -> float}

    où alpha et beta sont les mémoires respectives du meilleur coup ami ou ennemi observé dans l’arbre.
    Le reste des paramètres est similaire à ceux de Minimax (F.52). \\
    \arrayrulecolor{MediumAquamarine}\hline
    \arrayrulecolor{CornflowerBlue}
    La stratégie de validation est elle aussi similaire à celle de Minimax. On compare en profondeur 1 le score envoyé par AlphaBeta avec le score théorique qu’il est censé renvoyer pour un board prédéfini.

    Pour des profondeurs plus grandes, nous avons fait jouer AlphaBeta contre le joueur random en s’assurant que l’IA gagne la majorité du temps. \\
    \arrayrulecolor{MediumAquamarine}\hline
    \arrayrulecolor{CornflowerBlue}
    L'implémentation des heuristiques (F.51) est nécessaire pour l'évaluation des plateau dans l'algorithme alphabeta-pruning.\\
    Alphabeta est nécessaire pour implémenter le mode IA (F.24) \\
    \hline
\end{tabularx}

\vspace{1cm}

\noindent
\setlength{\arrayrulewidth}{1.5pt}
\renewcommand{\arraystretch}{1.5}
\arrayrulecolor{CornflowerBlue}
\begin{tabularx}{\textwidth}{|X|}
    \hline
    \textbf{F.56 Profondeur de recherche} \\
    \hline
    Besoin fonctionnel \\
    \hline
    Terminé le 25/03 \\
    Par Rémy \\
    \hline
    L’option \texttt{--ai-depth [DEPTH]} permet de spécifier une limite de profondeur de recherche au joueur IA.

    Par défaut, la profondeur maximale est fixée à 3.

    La taille de l’arbre de recherche construit par le joueur IA dans sa recherche du meilleur coup ne dépasse pas la profondeur maximale spécifiée. \\
    \arrayrulecolor{MediumAquamarine}\hline
    \arrayrulecolor{CornflowerBlue}
    Les algorithmes de recherche ont déjà été testés indépendamment à des profondeurs différentes, l’option \texttt{--ai-depth [DEPTH]} a donc simplement été testée en essayant de jouer contre une IA et en observant le mode debug. \\
    \arrayrulecolor{MediumAquamarine}\hline
    \arrayrulecolor{CornflowerBlue}
    L'implémentation des algorithmes minimax (F.52) et AlphaBeta (F.53) est nécessaire pour l'ajout de cette option. \\
    L'option \texttt{--ai-depth [DEPTH]} est une dépendance du mode IA (F.24) \\
    \hline
\end{tabularx}

\vspace{1cm}

\noindent
\setlength{\arrayrulewidth}{1.5pt}
\renewcommand{\arraystretch}{1.5}
\arrayrulecolor{CornflowerBlue}
\begin{tabularx}{\textwidth}{|X|}
    \hline
    \textbf{F.57 Choix des heuristiques} \\
    \hline
    Besoin fonctionnel \\
    \hline
    Terminé le 18/03 \\
    Par Rémy \\
    \hline
    L’option \texttt{--ai-heuristic [HEURISTIC]} permet de spécifier l’heuristique utilisée par l’IA pour évaluer un coup.

    Les heuristiques disponibles sont expliquées dans le besoin F.51. \\
    \arrayrulecolor{MediumAquamarine}\hline
    \arrayrulecolor{CornflowerBlue}
    Les heuristiques ont déjà été testées individuellement. Il suffisait donc de lancer des parties avec les différentes IA et vérifier qu’elles jouaient en prenant en compte la bonne caractéristique (les coins, le nombre de pierres, le nombre de coups possibles ou toutes à la fois). \\
    \arrayrulecolor{MediumAquamarine}\hline
    \arrayrulecolor{CornflowerBlue}
    L'implémentation de plusieurs heuristiques (F.51) est nécessaire pour  \\
    Dépendances inversées \\
    L'option \texttt{--ai-heuristic [HEURISTIC]} est une dépendance du mode IA (F.24) \\
    \hline
\end{tabularx}

\vspace{1cm}



\newpage

\section{Liste des besoins non réalisés}

Au début du projet, il semblait peu probable que nous réussissions à réaliser
tous les besoins du cahier des charges dans les temps.\\ Nous avons donc écarté
certains besoins qui nous paraissaient trop chronophages ou trop spécifiques,
et dont l’absence n'empêcherait pas le bon fonctionnement de notre logiciel.\\
Nous avons alors supprimé l’internationalisation (F.13), notre logiciel étant
uniquement en langue anglaise, ainsi que le mode verbose (F.18).\\ Nous avons
décidé d'avoir un historique des coups joués, mais de ne pas implémenter les
fonctions permettant d’annuler ou de rejouer des coups, ni de rejouer une
partie à partir d’un historique (besoins F.34 et F.37).\\ Pour l'interface
graphique, nous n’avons pas de menus (F.45, F.46, F.47), et il n’est pas
possible non plus d’éditer la configuration de la partie depuis l’interface
graphique (F.50).\\ Enfin, pour le mode IA, nous avons décidé de mettre de côté
l’implémentation du MCTS (Monte Carlo Tree Search), ainsi que le calcul des
pierres stables (besoins F.55, F.59).\\

Parmi les besoins que nous avions prévu d’implémenter, mais qui ne figurent pas
dans la version définitive de notre projet, il y a l’implémentation du mode
contest (F.23) et l’exploration peu profonde préliminaire (F.54).\\ Pour
implémenter l’option \texttt{--ai-shallow} (F.54), il fallait implémenter une
fonction \texttt{alphabeta with iterative deepening}. Cet algorithme est très
similaire à \texttt{alphabeta} classique, mais il introduit une variable de
profondeur actuelle qui est différente de la profondeur maximale à atteindre.
Le but de cet algorithme est donc d’effectuer un \texttt{alphabeta} de la
profondeur 1 à \texttt{n} (\texttt{n} étant la profondeur max). Après
l’exploration de la profondeur actuelle terminée, on ordonne les coups en
fonction du score renvoyé avant de poursuivre avec l’exploration à la
profondeur actuelle~+~1. Ce qui a pour but de maximiser le nombre de branches
élaguées par l’algorithme \texttt{alphabeta}.\\ Afin de stocker les scores
renvoyés à chaque profondeur, il fallait aussi implémenter une technique de
hashing pour stocker des \texttt{boards} (1 hash = 1 \texttt{board state}, ce
qui enlève les doublons) et une \texttt{hashtable}. La technique de hash
utilisée était la méthode Zobrist, qui consiste à attribuer un nombre binaire
unique pour chaque coup, et faire des \texttt{xor} des coups joués sur le hash
du board de départ pour obtenir de nouveaux hash pour chaque \texttt{board}.\\
À cause du manque de temps et des bugs/coverage à gérer en priorité, nous avons
décidé de rejeter ce besoin pour le bien du projet, et donc de supprimer cette
version de l’algorithme qui ne marchait pas.\\ L’option du temps borné (F.60)
dépend directement de \texttt{iterative deepening}. Cet algorithme enregistre
le meilleur coup à jouer pour chaque profondeur, et le met à jour à chaque
boucle. Il est alors possible de borner le temps de l’algo et ainsi stopper les
boucles avant que l’algorithme ait atteint la profondeur maximale.\\ Étant
donné l’annulation de \texttt{--ai-shallow} (F54), il est donc impossible
d’implémenter cette option.\\

Et les besoins suivants ne sont pas complets, ou ont été modifiés par rapport à
leur description donnée dans le cahier des charges.\\ La représentation de
l’historique (F.29) n’est pas complète, nous n’avons pas implémenté la
possibilité de commenter un coup avec du texte entre accolades.\\ L’affichage
de messages à l’utilisateur (F.31) se fait très bien en interface en lignes de
commande, mais dans l’interface graphique, nous n’avons pas de messages
affichés au joueur.\\ Seuls les timers des deux joueurs en mode blitz sont
présents, ainsi que la validation pour des commandes telles que l’abandon de la
partie.\\ Un coup invalide n’est simplement pas pris en compte, il n’y a pas de
message pour indiquer au joueur que là où il a cliqué n’est pas un coup
jouable.\\ Il n’y a pas la possibilité d’afficher les règles du jeu ni une
aide.\\

\newpage

\section{Agenda rétrospectif}

Nous avons décidé d’ajouter en annexe un agenda rétrospectif afin de voir à
quel point nous nous sommes écartés de l’agenda prévisionnel construit au tout
début du projet. En effet, nous avions une planification bancale à l’origine.
Certains besoins étaient à implémenter bien plus tard qu’ils n’étaient
nécessaires en réalité. Nous n’avions donc pas les besoins nécessaires pour
avoir un jeu jouable en un temps raisonnable — après la deuxième semaine du
projet.\\ Nous avons suivi l’agenda prévisionnel la première et deuxième
semaine, puis, nous rendant compte de notre erreur, nous avons réfléchi
ensemble à quoi nous avions besoin dans l’immédiat.\\ Nous avons séparé les
besoins — ou plutôt des groupements de besoins — en degrés d’importance,
priorisant un jeu en mode normal fonctionnel. En besoins moins urgents, nous
avons mis le mode Blitz et l’affinage de l’interface en lignes de commandes.
Enfin, en besoins qui pouvaient attendre avec le mode IA et l’interface
graphique.\\ C’est pourquoi l’agenda rétrospectif présenté ici ne respecte pas
le premier agenda que nous vous avions soumis.\\

\begin{figure}[h]

    \vspace{3cm}

    \centering
    \includegraphics[width=0.9\textwidth]{images/AR_week_01.png}
    \caption{Tâches effectuées pendant la première semaine du projet}
    \title{fig:AR_week_01.png}

    \vspace{1cm}

\end{figure}

\begin{figure}[h]
    \centering
    \includegraphics[width=0.9\textwidth]{images/AR_week_02.png}
    \caption{Tâches effectuées pendant la seconde semaine du projet}
    \title{fig:AR_week_0.png}

    \vspace{1cm}

    \includegraphics[width=0.9\textwidth]{images/AR_week_03.png}
    \caption{Tâches effectuées pendant la troisième semaine du projet}
    \title{fig:AR_week_0.png}
\end{figure}

\begin{figure}[h]
    \centering
    \includegraphics[width=0.9\textwidth]{images/AR_week_04.png}
    \caption{Tâches effectuées pendant la quatrième semaine du projet}
    \title{fig:AR_week_0.png}

    \vspace{1cm}

    \includegraphics[width=0.9\textwidth]{images/AR_week_05.png}
    \caption{Tâches effectuées pendant la cinquième semaine du projet}
    \title{fig:AR_week_0.png}

    \vspace{1cm}

    \includegraphics[width=0.9\textwidth]{images/AR_week_06.png}
    \caption{Tâches effectuées pendant la sixième semaine du projet}
    \title{fig:AR_week_0.png}
\end{figure}

\begin{figure}[h]
    \includegraphics[width=0.9\textwidth]{images/AR_week_07.png}
    \caption{Tâches effectuées pendant la septième semaine du projet}
    \title{fig:AR_week_0.png}

    \vspace{1cm}

    \includegraphics[width=0.9\textwidth]{images/AR_week_08-09.png}
    \caption{Tâches effectuées pendant les huitième et neuvième semaines du projet}
    \title{fig:AR_week_0.png}
\end{figure}

\FloatBarrier

\newpage

\section{Bibliographie}

\bibliographystyle{plainnat}
\bibliography{bibliography}

\end{document}

% \begin{lstlisting}[language=Python, caption=Déclaration de la fonction create_parser]
%     create_parser() -> ArgumentParser
% \end{lstlisting}
