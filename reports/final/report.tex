\documentclass[a4paper,12pt]{article}
\usepackage[utf8]{inputenc}
\usepackage[T1]{fontenc}
\usepackage[french]{babel}
\usepackage{csquotes}
\usepackage{geometry}
\usepackage{graphicx}
\usepackage{listings}
\usepackage{hyperref}
\usepackage{float}
\usepackage{appendix}
\usepackage{xcolor}
\usepackage{colortbl}
\usepackage{tabularx}
\usepackage{array}
\usepackage[utf8]{inputenc}
\usepackage[T1]{fontenc}
\usepackage[backend=biber]{biblatex}
\geometry{top=2cm, bottom=2cm, left=2cm, right=2cm}

\definecolor{CornflowerBlue}{RGB}{100,149,237}  % For functional needs
\definecolor{RoyalBlue}{RGB}{65,105,225}   % For non-functional needs
\definecolor{MediumAquamarine}{RGB}{102,205,170} % For validation strategy


\addbibresource{bibliography.bib}

\begin{document}

\begin{titlepage}
    \centering
    \vspace*{1cm}
    {\huge\bfseries Jeu de plateau: Othello}
    \vspace{3cm}

    {Matis Duval, Rémy Heuret, Lucas Marques, Gabriel Tardiou}
    \vspace{2cm}

    {\scshape\small Université de Bordeaux\\}
    \vspace{1cm}
    {\scshape\small Projet de Programmation, Master 1}
    \vspace{1cm}

    \vfill
    {\large Avril 2025}
\end{titlepage}

\newpage

\tableofcontents

\newpage

\section{Présentation de l'existant}

Le jeu d’Othello que l’on connaît aujourd’hui est une version modifiée du jeu
Reversi, créé autour de 1880, supposément par Lewis Waterman ou John W.
Mollett. Reversi lui-même étant une variante du jeu Annexation créé en 1870 par
J. W. Mollett – la seule différence étant le plateau de jeu: une croix de 10
par 4 pour Annexation, et un carré de 8 par 8 pour Reversi. Ces deux personnes
d’origine anglaise se disputent l’invention du jeu, et ont créé deux versions
différentes : Reversi Waterman\footnote{Image prise sur le site de la
    Fédération Française d’Othello.
    \url{https://www.ffothello.org/images/histoire/jeux-anciens/reversi_waterman-1880.jpg}}
et Reversi Mollett\footnote{Image prise sur le site de la Fédération Française
    d’Othello.
    \url{https://www.ffothello.org/images/histoire/jeux-anciens/reversi_mollett-1880.jpg}}.\newline

\begin{figure}[h]
    \centering
    \includegraphics[width=0.5\textwidth]{images/reversi_waterman-1880.jpg}
    \caption{Jeu Reversi, version Lewis Waterman .\\
        Le jeu comporte un papier représentant le plateau, un livret de règles, et des pions.}
    \title{fig:Jeu Reversi, version Lewis Waterman.}

    \centering
    \includegraphics[width=0.5\textwidth]{images/reversi_mollett-1880.jpg}
    \caption{Jeu Reversi, version J. W. Mollett.\\
        Le jeu comporte un plateau en papier et des pions.}
    \title{fig:Jeu Reversi, version J. W. Mollett.}
\end{figure}

Ce jeu de plateau était très apprécié vers la fin du 19\up{ème} siècle, surtout
en Angleterre. Un article sur le jeu parut en 1888 dans un magazine spécialisé
dans les jeux de dames.\newline On retrouve des éditions aux États-Unis, ainsi
qu’en Europe Centrale, et des livres de stratégie sont créés.\newline Le jeu
Reversi perd de sa popularité au 20\up{ème} siècle, jusqu’en 1971, où un
Japonais du nom de Goro Hasegawa réinvente et redistribua le jeu sous un autre
nom : Othello. Le père de Goro Hasegawa, un professeur de littérature anglaise,
lui proposa le nom d’Othello en référence à la pièce de W. Shakespeare, en
raison des nombreux retournements de situation.\newline Le jeu devient
rapidement populaire, et la première compétition est organisée en 1973, soit
deux ans après la commercialisation du jeu.\newline Dès 1976, Othello est
arrivé en Angleterre et aux États-Unis, et les premiers championnats du monde
d’Othello se tiennent en 1977, et reviennent tous les ans.\newline Les règles
d’Othello diffèrent légèrement de celles de Reversi. Désormais, on fixe la
position initiale des pions, et il est possible de prendre des pions à son
adversaire lorsque celui-ci passe son tour.\newline En France, le jeu se
popularise à partir de la fin des années 1970, et la Fédération Française
d’Othello (FFO) est créée en 1983\footnote{Image prise sur le site de la
    Fédération Française d’Othello.
    \url{https://www.ffothello.org/images/histoire/jeux-modernes/othelloroyal.jpg}}.\newline

\begin{figure}[h]
    \centering
    \includegraphics[width=0.5\textwidth]{images/othelloroyal.jpg}
    \caption{Jeu Othello, Othello Royal distribué par Tsukada.\\
        Jeu utilisé en tournois en France.}
    \title{fig:Jeu Reversi, version J. W. Mollett.}
\end{figure}

Afin d’établir des stratégies, les joueurs doivent avoir en tête certains
concepts clés. Tout d’abord, il faut se rappeler que le nombre de pièces de
chaque joueur peut changer rapidement au cours de la partie. Il est important
de placer des pierres stables, qui ne pourront pas être capturées.\newline
Ensuite, placer ses pions dans les coins garantit souvent les pièces autour
comme stables. Le joueur essaiera de ne pas jouer dans les cases proches des
coins pour ne pas les donner à son adversaire.\newline Il faut essayer de
réduire le plus possible les options de son adversaire, tout en essayant
d’avoir beaucoup de choix de son côté.\newline Calculer le nombre de cases
restant permet de se projeter sur la fin de partie, et de qui des deux joueurs
va placer le dernier pion. Comme le résultat est calculé à partir du dernier
état du jeu, jouer en dernier peut faire une différence. On peut calculer ses
prochains coups en conséquence, essayer de faire passer son tour à son
adversaire, ou l’inciter à jouer sur certaines cases.\newline

Les règles en tournoi, spécifiquement, pour les championnats du monde d’Othello
(World Othello Championships – WOC) sont décrites dans le document World
Othello Championships Rules, publié en septembre 2019 par la Fédération
Mondiale d’Othello (World Othello Federation – WOF).\newline Les championnats
du monde se tiennent annuellement, et déterminent le champion des catégories
Individuels, Femmes, et Jeunes.\newline La compétition se tient sur trois
jours. Pendant les deux premiers jours, le rang est déterminé par des matchs à
temps limité, en rondes Suisses ou Robin, puis les demi-finales et la finale se
tiennent le troisième jour.\newline Seulement les équipes des nations membres
de la WOF peuvent participer.\newline Les règles sont très spécifiques quant à
l’éligibilité des joueurs, le compte des scores, et l’organisation du tournoi.
Les procédures pour l’édition des listes de classement, ou en cas de matériel
défectueux, d’égalité y sont également décrites très précisément.\newline Le
site de la Fédération Mondiale d’Othello tient à jour un calendrier
\footnote{Calendrier international des tournois d'Othello, régulièrement mis à
    jour par la Fédération Mondiale d'Othello.
    \url{https://www.worldothello.org/calendar}} des prochains tournois
organisés.\newline Des championnats européens se tiendront le 31 mai et 1er
juin de cette année, à Prague en République Tchèque.\newline Le prochain
championnat du monde aura lieu à Ankara en Turquie, en novembre 2025, et
réunira plus de 84 pays.\newline

Le jeu d’Othello est relativement populaire, on en trouve fréquemment en clubs
de jeux de société ou jeux de plateau.\newline

Il est également possible de jouer en ligne sur des sites, recommandés par la
FFO \footnote{Guide des plateformes en ligne pour jouer à Othello, recommandé
    par la Fédération Française d'Othello.
    \url{https://www.ffothello.org/communaute/jouer-sur-internet/}}. Les plus
populaires cités sont PlayOK \footnote{Site PlayOK offrant la possibilité de
    jouer au Reversi en ligne contre des adversaires du monde entier, avec système
    de classement. \url{https://www.playok.com/fr/reversi/}}, qui propose des
parties en ligne ; et eOthello \footnote{Plateforme eOthello permettant de
    jouer gratuitement à Othello en différé. \url{https://www.eothello.com/}} pour
jouer plusieurs parties en différé, avec 72 heures de temps limite par
coup.\newline

Le site PlayPager \footnote{Plateforme PlayPager permettant de jouer
    gratuitement à Othello/Reversi en ligne contre l'ordinateur ou d'autres
    joueurs. \url{https://playpager.com/othello-reversi/}} permet de jouer à deux
joueurs ou contre une intelligence artificielle sur un plateau en 8 par
8.\newline La Fédération Française d’Othello propose également de se mettre en
relation avec des joueurs via une liste de diffusion.\newline Il existe aussi
des serveurs discord tels que Othello Academy \footnote{Serveur Discord
    "Othello Academy" regroupant une communauté active de joueurs, des tutoriels et
    des tournois amicaux. \url{https://discord.me/othelloacademy}}, ou encore Board
Games Geek.\newline

\newpage

\section{Règles du jeu}

\section{Descriptions des algorithmes et structures de données}

\newpage

\section{Architecture du projet}

\newpage

\section{Performances et limitations}

\section{Critiques et perspectives}

\newpage

\section{Bibliographie}

\newpage
\appendix
\section{Liste des besoins étendus}


% For a functional need - border in CornflowerBlue
\noindent
\setlength{\arrayrulewidth}{1.5pt}
\renewcommand{\arraystretch}{1.5}
\arrayrulecolor{CornflowerBlue}
\begin{tabularx}{\textwidth}{|X|}
\hline
\textbf{F.xx NOM DU BESOIN} \\
\hline
Besoin fonctionnel\\
\hline
Terminé le DATE\\
Par NOM \\
\hline
Explication du\\
besoin \\
\arrayrulecolor{MediumAquamarine}\hline
\arrayrulecolor{CornflowerBlue}
Stratégie de validation \\
\arrayrulecolor{MediumAquamarine}\hline
\arrayrulecolor{CornflowerBlue}
Dépendances \\
\hline
\end{tabularx}

\vspace{1cm}

% For a non-functional need (RoyalBlue)
\noindent
\setlength{\arrayrulewidth}{1.5pt}
\renewcommand{\arraystretch}{1.5}
\arrayrulecolor{RoyalBlue}
\begin{tabularx}{\textwidth}{|X|}
\hline
\textbf{F.xx NOM DU BESOIN} \\
\hline
Besoin non fonctionnel\\
\hline
Terminé le DATE\\
Par NOM \\
\hline
Explication du\\
besoin \\
\arrayrulecolor{MediumAquamarine}\hline
\arrayrulecolor{RoyalBlue}
Stratégie de validation \\
\arrayrulecolor{MediumAquamarine}\hline
\arrayrulecolor{RoyalBlue}
Dépendances \\
\hline
\end{tabularx}

\vspace{1cm}


\newpage

\noindent
\setlength{\arrayrulewidth}{1.5pt}
\renewcommand{\arraystretch}{1.5}
\arrayrulecolor{RoyalBlue}
\begin{tabularx}{\textwidth}{|X|}
\hline
\textbf{F.1 Langage de programmation} \\
\hline
Besoin non fonctionnel\\
\hline
Terminé le 06/02\\
Par Lucas \\
\hline
Développer le logiciel en langage Python 3.7+.\\
\hline
\end{tabularx}

\vspace{1cm}

\noindent
\setlength{\arrayrulewidth}{1.5pt}
\renewcommand{\arraystretch}{1.5}
\arrayrulecolor{RoyalBlue}
\begin{tabularx}{\textwidth}{|X|}
\hline
\textbf{F.2 - Style de codage} \\
\hline
Besoin non fonctionnel\\
\hline
Terminé le 04/02\\
Par Lucas \\
\hline
Le code du programme suit le coding style du PEP8.\\
L’utilisation de pre-commit hooks permet de s’assurer du respect de ce style de codage.\\
\hline
\end{tabularx}

\vspace{1cm}

\noindent
\setlength{\arrayrulewidth}{1.5pt}
\renewcommand{\arraystretch}{1.5}
\arrayrulecolor{RoyalBlue}
\begin{tabularx}{\textwidth}{|X|}
\hline
\textbf{F.3 - Langue par défaut du code} \\
\hline
Besoin non fonctionnel\\
\hline
Terminé le 02/02\\
Par Matis, Rémy, Lucas, Gabriel \\
\hline
Le code, la documentation ainsi que les fichiers annexes sont écrits et commentés intégralement en langue anglaise.\\
\hline
\end{tabularx}

\vspace{1cm}

\noindent
\setlength{\arrayrulewidth}{1.5pt}
\renewcommand{\arraystretch}{1.5}
\arrayrulecolor{RoyalBlue}
\begin{tabularx}{\textwidth}{|X|}
\hline
\textbf{F.4 - Système cible
} \\
\hline
Besoin non fonctionnel\\
\hline
Terminé le 06/02\\
Par Lucas Zammit \\
\hline
Développement du logiciel sous une distribution Linux.\\
L’intégration de Python rend compliqué voir impossible l’installation de modules
python sur l’environnement par défaut.\\
Sur les systèmes Debian récents, il est donc nécessaire de passer par un environnement virtuel.\\
\arrayrulecolor{MediumAquamarine}\hline
\arrayrulecolor{RoyalBlue}
Tester sous une distribution Linux Debian, et une distribution Ubuntu. \\
\hline
\end{tabularx}

\vspace{1cm}

\noindent
\setlength{\arrayrulewidth}{1.5pt}
\renewcommand{\arraystretch}{1.5}
\arrayrulecolor{RoyalBlue}
\begin{tabularx}{\textwidth}{|X|}
\hline
\textbf{F.5 - Documentation} \\
\hline
Besoin non fonctionnel\\
\hline
Terminé le DATE\\
Par Matis, Rémy, Lucas, Gabriel \\
\hline
Une documentation Sphinx automatisée permet de rendre compte des différentes fonctions, leurs attributs et leur utilité pour chaque module.\\
Le code est documenté à l’aide de docstrings et commentaires.\\
Un manuel utilisateur, ainsi qu’un manuel utilisateur pour les développeurs sont disponibles. Ils proposent des explications pour l’installation et l’utilisation du logiciel d’un point de vue utilisateur, et en proposant plus de précisions pour les développeurs.\\
Une option d’aide est présente en dehors d’une partie, afin d’indiquer à l’utilisateur comment lancer une partie et quelles options sont à sa disposition.\\
En cours de partie, une option d’aide et une présentation des règles du jeu est également disponible.\\
\hline
\end{tabularx}

\vspace{1cm}

\noindent
\setlength{\arrayrulewidth}{1.5pt}
\renewcommand{\arraystretch}{1.5}
\arrayrulecolor{RoyalBlue}
\begin{tabularx}{\textwidth}{|X|}
\hline
\textbf{F.6 - Tests} \\
\hline
Besoin non fonctionnel\\
\hline
Terminé le DATE\\
Par Matis, Rémy, Lucas, Gabriel \\
\hline
Les fonctionnalités du logiciel sont testées automatiquement. Il est possible de les lancer avec la commande \texttt{pytest} à la racine du projet ou dans le dossier othello/.\\
L’automatisation des tests est intégrée à la pipeline CI gitlab.\\
Ces tests couvrent au minimum 80\% du code écrit.\\
\hline
\end{tabularx}

\vspace{1cm}
\noindent
\setlength{\arrayrulewidth}{1.5pt}
\renewcommand{\arraystretch}{1.5}
\arrayrulecolor{RoyalBlue}
\begin{tabularx}{\textwidth}{|X|}
\hline
\textbf{F.7 - Bugs} \\
\hline
Besoin non fonctionnel\\
\hline
Terminé le DATE\\
Par Matis, Rémy, Lucas, Gabriel \\
\hline
L’implémentation des fonctionnalités du logiciel seront robustes - dans la mesure du possible - aux éventuelles mauvaises entrées de l’utilisateur, ou à d’autres cas amenant à des bugs.\\
\arrayrulecolor{MediumAquamarine}\hline
\arrayrulecolor{RoyalBlue}
Afin de pallier d'éventuels bugs menant à un crash du logiciel, les fonctionnalités sont testées le plus possible, et les erreurs d’entrées ou d’arguments seront traitées et renverront une erreur compréhensible à l’utilisateur.\\
\hline
\end{tabularx}

\vspace{1cm}

\noindent
\setlength{\arrayrulewidth}{1.5pt}
\renewcommand{\arraystretch}{1.5}
\arrayrulecolor{RoyalBlue}
\begin{tabularx}{\textwidth}{|X|}
\hline
\textbf{F.8 - Performances} \\
\hline
Besoin non fonctionnel\\
\hline
Terminé le DATE\\
Par NOM \\
\hline
Explication du\\
besoin \\
\arrayrulecolor{MediumAquamarine}\hline
\arrayrulecolor{RoyalBlue}
Stratégie de validation \\
\arrayrulecolor{MediumAquamarine}\hline
\arrayrulecolor{RoyalBlue}
Dépendances \\
\hline
\end{tabularx}

\vspace{1cm}

\noindent
\setlength{\arrayrulewidth}{1.5pt}
\renewcommand{\arraystretch}{1.5}
\arrayrulecolor{RoyalBlue}
\begin{tabularx}{\textwidth}{|X|}
\hline
\textbf{F.9 - Build-system} \\
\hline
Besoin non fonctionnel\\
\hline
Terminé le 04/02\\
Par Rémy \\
\hline
Le build-system est basé sur pyproject.toml et septuptools.\\
\hline
\end{tabularx}

\vspace{1cm}

\noindent
\setlength{\arrayrulewidth}{1.5pt}
\renewcommand{\arraystretch}{1.5}
\arrayrulecolor{RoyalBlue}
\begin{tabularx}{\textwidth}{|X|}
\hline
\textbf{F.10 - Framework de tests} \\
\hline
Besoin non fonctionnel\\
\hline
Terminé le 04/02\\
Par Matis et Lucas \\
\hline
Mise en place de la pipeline CI/CD dans git.
Avant chaque commit, les tests seront lancés par la pipeline.\\
Utilisation du framework de tests Pytest et de Coverage.\\
\hline
\end{tabularx}

\vspace{1cm}

\noindent
\setlength{\arrayrulewidth}{1.5pt}
\renewcommand{\arraystretch}{1.5}
\arrayrulecolor{CornflowerBlue}
\begin{tabularx}{\textwidth}{|X|}
\hline
\textbf{F.11 - Gestion des options} \\
\hline
Besoin fonctionnel\\
\hline
Terminé le 18/02\\
Par Gabriel \\
\hline
Les options en lignes de commandes pour le lancement d’une partie sont gérées par le module argparse qui construit un objet contenant les arguments donnés par l’utilisateur.\\
Les arguments passés au programme prennent systématiquement la priorité sur les
paramètres définis dans le fichier de configuration othello, qui lui-même aura la priorité
sur les paramètres par défaut.\\

Les fonctions principales implémentées sont les suivantes :\\
\texttt{create\_parser() -> ArgumentParser}, qui initialise et configure le parseur d’arguments ;\\
\texttt{parse\_args() -> tuple[str, dict[str, Any]]}, qui parse les arguments, et gère les potentielles erreurs d’entrée : si l’utilisateur a entré de mauvais arguments, ou une option non reconnue ; et\\
\texttt{parse\_error(parser: ArgumentParser, message: str) -> None}, qui affiche des messages d’erreurs dans la console, en utilisant la sortie d’erreur standard.\\

\arrayrulecolor{MediumAquamarine}\hline
\arrayrulecolor{CornflowerBlue}
Validation du parseur avec une configuration par défaut, avec un fichier de configuration, et avec les différentes options disponibles.\\
Validation de la gestion d’erreurs avec des configurations invalides : option non existante ou dont la valeur donnée n’est pas acceptable, incompatibilité des modes de jeu.\\
\arrayrulecolor{MediumAquamarine}\hline
\arrayrulecolor{CornflowerBlue}
Nécessite la définition d’un fichier de configuration (F.40), afin de connaître toutes les options que le parseur doit proposer à l’utilisateur pour le lancement d’une partie.\\
Nécessaire pour lancer le jeu (F.15), ainsi que les différents modes de jeu et l’affichage de messages à l’utilisateur (besoins F.16 à F.24).\\
\hline
\end{tabularx}

\vspace{1cm}

\noindent
\setlength{\arrayrulewidth}{1.5pt}
\renewcommand{\arraystretch}{1.5}
\arrayrulecolor{RoyalBlue}
\begin{tabularx}{\textwidth}{|X|}
\hline
\textbf{F.12 - Bibliothèque graphique} \\
\hline
Besoin non fonctionnel\\
\hline
Terminé le 27/02\\
Par Lucas \\
\hline
Utilisation de la bibliothèque graphique PyGObject.\\
Utilisation de la version 4.0 de Gtk.\\
\hline
\arrayrulecolor{RoyalBlue}
Nécessaire pour l’implémentation de l’interface graphique (F.26).\\
\hline
\end{tabularx}

\vspace{1cm}

\noindent
\setlength{\arrayrulewidth}{1.5pt}
\renewcommand{\arraystretch}{1.5}
\arrayrulecolor{RoyalBlue}
\begin{tabularx}{\textwidth}{|X|}
\hline
\textbf{F.14 - Nom du programme} \\
\hline
Besoin non fonctionnel\\
\hline
Terminé le 09/02\\
Par Lucas \\
\hline
Le logiciel pourra être lancé avec l’exécutable ‘othello’.\\
Taper ‘othello’ dans l’invite de commande permet de lancer une partie en mode normal, joueur contre joueur, sur un plateau de huit cases.\\
Il sera nécessaire de lancer le programme depuis un environnement virtuel.\\
\hline
\end{tabularx}

\vspace{1cm}

\noindent
\setlength{\arrayrulewidth}{1.5pt}
\renewcommand{\arraystretch}{1.5}
\arrayrulecolor{CornflowerBlue}
\begin{tabularx}{\textwidth}{|X|}
\hline
\textbf{F.15 - Usage général} \\
\hline
Besoin fonctionnel\\
\hline
Terminé le 18/02\\
Par Matis et Lucas \\
\hline
Lancement du programme principal pour jouer une partie d’Othello avec la commande suivante :\\
\texttt{othello [OPTIONS] [FILENAME]}\\
Pour cela, nous aurons besoin d’une initialisation de la partie.\\
D’abord, récupérer les options données par l’utilisateur, et en récupérer une configuration pour la partie.\\
Récupérer une partie précédente depuis un fichier si précisé.\\
Suivant le mode de jeu, lancer la boucle de jeu correspondante.\\
Et afficher le jeu en interface graphique ou en lignes de commandes.\\
\hline
Nécessite de lancer le programme avec le nom de l’exécutable (F.14).\\
Nécessite l’implémentation du parser (F.11), et des boucles de jeu pour un mode normal, blitz (F.21) et IA (F.24).\\
Ainsi que l’implémentation de la CLI (F.25) et de la GUI (F.26).\\
\hline
\end{tabularx}

\vspace{1cm}

\noindent
\setlength{\arrayrulewidth}{1.5pt}
\renewcommand{\arraystretch}{1.5}
\arrayrulecolor{CornflowerBlue}
\begin{tabularx}{\textwidth}{|X|}
\hline
\textbf{F.16 - Aide en ligne de commande} \\
\hline
Besoin fonctionnel\\
\hline
Terminé le 11/02 et 10/03\\
Par Gabriel, et Matis et Lucas\\
\hline
Affichage de l’aide pour lancer le programme avec les options \texttt{-h} ou \texttt{--help}.\\
Construction de l’aide du lancement du programme avec le parseur d’argument directement : utilisation de l’argument “help=’'” de argparse.\\

Affichage de l’aide et des règles du jeu durant la partie avec les options \texttt{?} et \texttt{r} respectivement.\\
Lorsque l’utilisateur entre une commande non reconnue en cours de partie, l’aide et des exemples de commandes sont affichés à la suite de l’entrée de l’utilisateur.\\
\arrayrulecolor{MediumAquamarine}\hline
\arrayrulecolor{CornflowerBlue}
Validation en vérifiant que les deux options pour afficher l’aide en dehors d’une partie affichent bien le message d’aide pour chaque option disponible.\\
Vérification que ces deux options font terminer le programme avec \texttt{EXIT\_SUCCESS}.\\
Validation en vérifiant que les messages d’aide et de règles sont bien affichés dans leur intégralité lors d’un appel à ces commandes pendant une partie.\
Vérification qu’après l’affichage de ces messages, la partie continue, en attendant toujours une commande de la part du même joueur.\\
\arrayrulecolor{MediumAquamarine}\hline
\arrayrulecolor{CornflowerBlue}
Nécessite une implémentation du parseur d’arguments incluant des messages d’aide personnalisés (F.11).\\
Nécessaire pour avoir une interface en ligne de commande complète (F.25).\\
\hline
\end{tabularx}

\vspace{1cm}

\noindent
\setlength{\arrayrulewidth}{1.5pt}
\renewcommand{\arraystretch}{1.5}
\arrayrulecolor{CornflowerBlue}
\begin{tabularx}{\textwidth}{|X|}
\hline
\textbf{F.17 - Version} \\
\hline
Besoin fonctionnel\\
\hline
Terminé le 11/02\\
Par Gabriel \\
\hline
Affichage de la version actuelle du logiciel avec les options \texttt{-v} ou \texttt{--version}.\\
\arrayrulecolor{MediumAquamarine}\hline
\arrayrulecolor{CornflowerBlue}
Validation en vérifiant que les deux options permettent bien d’afficher la version correcte.
Vérification que ces deux options font terminer le programme avec \texttt{EXIT\_SUCCESS}.\\
\arrayrulecolor{MediumAquamarine}\hline
\arrayrulecolor{CornflowerBlue}
Nécessite la définition de la version courante dans le fichier d’initialisation du logiciel.\\
\hline
\end{tabularx}

\vspace{1cm}

\noindent
\setlength{\arrayrulewidth}{1.5pt}
\renewcommand{\arraystretch}{1.5}
\arrayrulecolor{CornflowerBlue}
\begin{tabularx}{\textwidth}{|X|}
\hline
\textbf{F.19 - Mode debug} \\
\hline
Besoin fonctionnel\\
\hline
Terminé le 27/02\\
Par Gabriel \\
\hline
Activation du mode Debug avec les options \texttt{-d} ou \texttt{--debug} lors du lancement du programme.\\
Au cours de la partie, des messages de debug sont écrits dans un fichier “othello.log”.\\
Ces messages préviennent, entre autres, de l’état du jeu courant, des fonctions appelées, des entrées de l’utilisateur et de leur traitement, ainsi que des potentielles erreurs rencontrées, avec un contexte et un traçage de l’erreur.\\

Utilisation du module logging pour construire le logger, puis écriture de messages de debug pour pouvoir suivre les appels de fonctions et les erreurs en cours de partie.\\
Les fonctions principales implémentées sont les suivantes :\\
\texttt{logging\_config() -> None} qui configure le logger, en précisant le format d’affichage des messages, ainsi que le nom du fichier dans lequel écrire les messages de debug.\\
\texttt{log\_error\_message() -> None} qui affiche des messages d’erreurs customisés, avec un contexte affiché s'il est fourni, et un traçage de l’erreur relevée.\\
\arrayrulecolor{MediumAquamarine}\hline
\arrayrulecolor{CornflowerBlue}
Validation en vérifiant que le logger est bien nommé “Othello”, et que l’on utilise toujours la même instance du logger pour tous les messages.\\
Vérification que si le mode Debug n’est pas activé par l’utilisateur, le logger ne s’initialise pas.\\
Validation en vérifiant que tous types de messages : informations de debug, erreurs avec et sans contexte soient bien écrits dans le fichier .log.\\
\arrayrulecolor{MediumAquamarine}\hline
\arrayrulecolor{CornflowerBlue}
Nécessaire pour avoir un affichage de messages pertinents pour un utilisateur versé dans la programmation (F.31).\\
\hline
\end{tabularx}

\vspace{1cm}

\noindent
\setlength{\arrayrulewidth}{1.5pt}
\renewcommand{\arraystretch}{1.5}
\arrayrulecolor{CornflowerBlue}
\begin{tabularx}{\textwidth}{|X|}
\hline
\textbf{F.20 - Taille du plateau} \\
\hline
Besoin fonctionnel\\
\hline
Terminé le 12/02\\
Par Lucas \\
\hline
Possibilité de jouer sur un Othellier de 6×6, 8×8, 10×10, ou 12×12 cases.\\
Représentations des différentes tailles de plateau avec les bitboards.\\
\arrayrulecolor{MediumAquamarine}\hline
\arrayrulecolor{CornflowerBlue}
Stratégie de validation \\
\arrayrulecolor{MediumAquamarine}\hline
\arrayrulecolor{CornflowerBlue}
Nécessite le début de l’implémentation des bitboards (F.42).\\
Nécessaire pour pouvoir jouer une partie : le plateau de base étant en 8 par 8 cases (F.15).\\
\hline
\end{tabularx}

\vspace{1cm}

\noindent
\setlength{\arrayrulewidth}{1.5pt}
\renewcommand{\arraystretch}{1.5}
\arrayrulecolor{CornflowerBlue}
\begin{tabularx}{\textwidth}{|X|}
\hline
\textbf{F.21 - Mode blitz} \\
\hline
Besoin fonctionnel\\
\hline
Terminé le 20/02\\
Par Matis \\
\hline
Les options \texttt{-b} ou \texttt{--blitz} permettent d’activer le mode blitz.\\
Mise en place de la boucle de jeu pour le mode blitz : changement des joueurs, gestion du temps au cours de la partie, et passage en texte des valeurs des timers.\\

Les fonctions principales implémentées sont les suivantes :\\
\texttt{change\_player(player: str) -> None} qui est utilisée pour mettre en oeuvre la rotation des joueurs.\\
\texttt{is\_time\_up(player: str) -> bool} qui renvoie l’état d’un timer, s’il reste encore du temps ou qu’il s’est totalement écoulé.\\

\arrayrulecolor{MediumAquamarine}\hline
\arrayrulecolor{CornflowerBlue}
Validation en vérifiant que le changement de joueur pause le timer du joueur courant, démarre le timer du nouveau joueur, et change le joueur courant pour le nouveau joueur.\\
Vérification que lorsque l’un des timer arrive à zéro, la fonction <is\_time\_up> renvoie Vrai, et Faux sinon.\\
\arrayrulecolor{MediumAquamarine}\hline
\arrayrulecolor{CornflowerBlue}
Nécessite l’implémentation des timers pour le mode blitz (F.22).\\
Nécessaire pour l’implémentation d’un jeu fonctionnel plus complet (F.15).\\
Nécessaire pour l’affichage du plateau (en CLI et GUI) (F.25, F.26, et F.27), ainsi que pour déterminer une fin de partie par faute de temps (F.36).\\
\hline
\end{tabularx}

\vspace{1cm}

\noindent
\setlength{\arrayrulewidth}{1.5pt}
\renewcommand{\arraystretch}{1.5}
\arrayrulecolor{CornflowerBlue}
\begin{tabularx}{\textwidth}{|X|}
\hline
\textbf{F.22 - Durée du blitz} \\
\hline
Besoin fonctionnel\\
\hline
Terminé le 12/02\\
Par Gabriel \\
\hline
Implémentation d’un timer pour le mode blitz.\\
Ajouter un temps \texttt{-t TIME} ou \texttt{--time TIME} permet d’imposer un temps en minutes, choisi par l’utilisateur, limitant la durée de la partie en mode blitz. Par défaut, le temps limite est fixé à 30 minutes par joueur.\\
Si le mode blitz n’est pas actif, cette option n’est pas considérée.\\

Les fonctions principales implémentées sont les suivantes:\\
\texttt{init(time\_limit: int) -> None} qui initialise les timers des deux joueurs à la valeur de time\_limit.\\
\texttt{start\_timer(player: str) -> None} qui démarre le timer du joueur spécifié, et indique que le joueur spécifié est le joueur courant.\\
\texttt{pause\_timer() -> None} qui stoppe le timer du joueur courant et update son temps restant.\\
\texttt{get\_remaining\_time(player: str) -> float} qui actualise le temps du joueur spécifié et le renvoie en secondes.\\
Les timers ne sont pas initialisés s’ils existent déjà. Il n’est pas possible de mettre un timer en pause s’il n’est pas en train de tourner dans un premier temps. Et le temps ne peut pas prendre de valeur négative, il est bloqué à zéro au minimum.\\
\arrayrulecolor{MediumAquamarine}\hline
\arrayrulecolor{CornflowerBlue}
Vérification des potentielles erreurs d’initialisation et de timer négatif.
Validation en vérifiant que les timers sont bien initialisés avec la valeur de \texttt{time\_limit} donnée, et que le joueur courant est fixé au joueur aux pions noirs.\\
Vérification que lorsque l’on actualise le timer, si la valeur est négative, on fixe le temps à zéro.\\
\arrayrulecolor{MediumAquamarine}\hline
\arrayrulecolor{CornflowerBlue}
Nécessaire pour l’implémentation du mode Blitz (F.21).\\
\hline
\end{tabularx}

\vspace{1cm}


\newpage

\section{Liste des besoins non réalisés}

\newpage

\section{Agenda rétrospectif}


Nous avons décidé d’ajouter en annexe un agenda rétrospectif afin de voir à
quel point nous nous sommes écartés de l’agenda prévisionnel construit au tout
début du projet. En effet, nous avions une planification bancale à l’origine.
Certains besoins étaient à implémenter bien plus tard qu’ils n’étaient
nécessaires en réalité. Nous n’avions donc pas les besoins nécessaires pour
avoir un jeu jouable en un temps raisonnable — après la deuxième semaine du
projet.\newline Nous avons suivi l’agenda prévisionnel la première et deuxième
semaine, puis, nous rendant compte de notre erreur, nous avons réfléchi
ensemble à quoi nous avions besoin dans l’immédiat.\newline Nous avons séparé
les besoins — ou plutôt des groupements de besoins — en degrés d’importance,
priorisant un jeu en mode normal fonctionnel. En besoins moins urgents, nous
avons mis le mode Blitz et l’affinage de l’interface en lignes de commandes.
Enfin, en besoins qui pouvaient attendre avec le mode IA et l’interface
graphique.\newline C’est pourquoi l’agenda rétrospectif présenté ici ne
respecte pas le premier agenda que nous vous avions soumis.\newline

\begin{figure}[h]
    \centering
    \includegraphics[width=0.9\textwidth]{images/AR_week_01.png}
    \caption{Exemple d'insertion d'image}
    \title{fig:AR_week_0.png}
\end{figure}

\begin{figure}[h]
    \centering
    \includegraphics[width=0.9\textwidth]{images/AR_week_02.png}
    \caption{Exemple d'insertion d'image}
    \title{fig:AR_week_0.png}

    \includegraphics[width=0.9\textwidth]{images/AR_week_03.png}
    \caption{Exemple d'insertion d'image}
    \title{fig:AR_week_0.png}
\end{figure}

\begin{figure}[h]
    \centering
    \includegraphics[width=0.9\textwidth]{images/AR_week_04.png}
    \caption{Exemple d'insertion d'image}
    \title{fig:AR_week_0.png}

    \includegraphics[width=0.9\textwidth]{images/AR_week_05.png}
    \caption{Exemple d'insertion d'image}
    \title{fig:AR_week_0.png}

    \includegraphics[width=0.9\textwidth]{images/AR_week_06.png}
    \caption{Exemple d'insertion d'image}
    \title{fig:AR_week_0.png}
\end{figure}

\newpage
\newpage

% Example of how to insert code
\begin{lstlisting}[language=C, caption=Exemple d'insertion de code]
    // Voici un exemple de code en C
    #include <stdio.h>

    int main() {
        printf("Hello, Othello!\n");
        return 0;
    }
\end{lstlisting}

% Example of how to insert an image
\begin{figure}[h]
    \centering
    \includegraphics[width=0.7\textwidth]{example-image}
    \caption{Exemple d'insertion d'image}
    \title{fig:example-image}
\end{figure}

% Example of a footnote
Voici un exemple de texte avec une note de bas de page\footnote{Ceci est un
    exemple de note de bas de page.}.

% Example of citation
Ceci est un exemple de citation \citation{Othello1980}.

\end{document}

% \begin{lstlisting}[language=Python, caption=Déclaration de la fonction create_parser]
%     create_parser() -> ArgumentParser
% \end{lstlisting}
